\documentclass[11pt]{article}

    \usepackage[breakable]{tcolorbox}
    \usepackage{parskip} % Stop auto-indenting (to mimic markdown behaviour)
    

    % Basic figure setup, for now with no caption control since it's done
    % automatically by Pandoc (which extracts ![](path) syntax from Markdown).
    \usepackage{graphicx}
    % Maintain compatibility with old templates. Remove in nbconvert 6.0
    \let\Oldincludegraphics\includegraphics
    % Ensure that by default, figures have no caption (until we provide a
    % proper Figure object with a Caption API and a way to capture that
    % in the conversion process - todo).
    \usepackage{caption}
    \DeclareCaptionFormat{nocaption}{}
    \captionsetup{format=nocaption,aboveskip=0pt,belowskip=0pt}

    \usepackage{float}
    \floatplacement{figure}{H} % forces figures to be placed at the correct location
    \usepackage{xcolor} % Allow colors to be defined
    \usepackage{enumerate} % Needed for markdown enumerations to work
    \usepackage{geometry} % Used to adjust the document margins
    \usepackage{amsmath} % Equations
    \usepackage{amssymb} % Equations
    \usepackage{textcomp} % defines textquotesingle
    % Hack from http://tex.stackexchange.com/a/47451/13684:
    \AtBeginDocument{%
        \def\PYZsq{\textquotesingle}% Upright quotes in Pygmentized code
    }
    \usepackage{upquote} % Upright quotes for verbatim code
    \usepackage{eurosym} % defines \euro

    \usepackage{iftex}
    \ifPDFTeX
        \usepackage[T1]{fontenc}
        \IfFileExists{alphabeta.sty}{
              \usepackage{alphabeta}
          }{
              \usepackage[mathletters]{ucs}
              \usepackage[utf8x]{inputenc}
          }
    \else
        \usepackage{fontspec}
        \usepackage{unicode-math}
    \fi

    \usepackage{fancyvrb} % verbatim replacement that allows latex
    \usepackage{grffile} % extends the file name processing of package graphics 
                         % to support a larger range
    \makeatletter % fix for old versions of grffile with XeLaTeX
    \@ifpackagelater{grffile}{2019/11/01}
    {
      % Do nothing on new versions
    }
    {
      \def\Gread@@xetex#1{%
        \IfFileExists{"\Gin@base".bb}%
        {\Gread@eps{\Gin@base.bb}}%
        {\Gread@@xetex@aux#1}%
      }
    }
    \makeatother
    \usepackage[Export]{adjustbox} % Used to constrain images to a maximum size
    \adjustboxset{max size={0.9\linewidth}{0.9\paperheight}}

    % The hyperref package gives us a pdf with properly built
    % internal navigation ('pdf bookmarks' for the table of contents,
    % internal cross-reference links, web links for URLs, etc.)
    \usepackage{hyperref}
    % The default LaTeX title has an obnoxious amount of whitespace. By default,
    % titling removes some of it. It also provides customization options.
    \usepackage{titling}
    \usepackage{longtable} % longtable support required by pandoc >1.10
    \usepackage{booktabs}  % table support for pandoc > 1.12.2
    \usepackage{array}     % table support for pandoc >= 2.11.3
    \usepackage{calc}      % table minipage width calculation for pandoc >= 2.11.1
    \usepackage[inline]{enumitem} % IRkernel/repr support (it uses the enumerate* environment)
    \usepackage[normalem]{ulem} % ulem is needed to support strikethroughs (\sout)
                                % normalem makes italics be italics, not underlines
    \usepackage{mathrsfs}
    

    
    % Colors for the hyperref package
    \definecolor{urlcolor}{rgb}{0,.145,.698}
    \definecolor{linkcolor}{rgb}{.71,0.21,0.01}
    \definecolor{citecolor}{rgb}{.12,.54,.11}

    % ANSI colors
    \definecolor{ansi-black}{HTML}{3E424D}
    \definecolor{ansi-black-intense}{HTML}{282C36}
    \definecolor{ansi-red}{HTML}{E75C58}
    \definecolor{ansi-red-intense}{HTML}{B22B31}
    \definecolor{ansi-green}{HTML}{00A250}
    \definecolor{ansi-green-intense}{HTML}{007427}
    \definecolor{ansi-yellow}{HTML}{DDB62B}
    \definecolor{ansi-yellow-intense}{HTML}{B27D12}
    \definecolor{ansi-blue}{HTML}{208FFB}
    \definecolor{ansi-blue-intense}{HTML}{0065CA}
    \definecolor{ansi-magenta}{HTML}{D160C4}
    \definecolor{ansi-magenta-intense}{HTML}{A03196}
    \definecolor{ansi-cyan}{HTML}{60C6C8}
    \definecolor{ansi-cyan-intense}{HTML}{258F8F}
    \definecolor{ansi-white}{HTML}{C5C1B4}
    \definecolor{ansi-white-intense}{HTML}{A1A6B2}
    \definecolor{ansi-default-inverse-fg}{HTML}{FFFFFF}
    \definecolor{ansi-default-inverse-bg}{HTML}{000000}

    % common color for the border for error outputs.
    \definecolor{outerrorbackground}{HTML}{FFDFDF}

    % commands and environments needed by pandoc snippets
    % extracted from the output of `pandoc -s`
    \providecommand{\tightlist}{%
      \setlength{\itemsep}{0pt}\setlength{\parskip}{0pt}}
    \DefineVerbatimEnvironment{Highlighting}{Verbatim}{commandchars=\\\{\}}
    % Add ',fontsize=\small' for more characters per line
    \newenvironment{Shaded}{}{}
    \newcommand{\KeywordTok}[1]{\textcolor[rgb]{0.00,0.44,0.13}{\textbf{{#1}}}}
    \newcommand{\DataTypeTok}[1]{\textcolor[rgb]{0.56,0.13,0.00}{{#1}}}
    \newcommand{\DecValTok}[1]{\textcolor[rgb]{0.25,0.63,0.44}{{#1}}}
    \newcommand{\BaseNTok}[1]{\textcolor[rgb]{0.25,0.63,0.44}{{#1}}}
    \newcommand{\FloatTok}[1]{\textcolor[rgb]{0.25,0.63,0.44}{{#1}}}
    \newcommand{\CharTok}[1]{\textcolor[rgb]{0.25,0.44,0.63}{{#1}}}
    \newcommand{\StringTok}[1]{\textcolor[rgb]{0.25,0.44,0.63}{{#1}}}
    \newcommand{\CommentTok}[1]{\textcolor[rgb]{0.38,0.63,0.69}{\textit{{#1}}}}
    \newcommand{\OtherTok}[1]{\textcolor[rgb]{0.00,0.44,0.13}{{#1}}}
    \newcommand{\AlertTok}[1]{\textcolor[rgb]{1.00,0.00,0.00}{\textbf{{#1}}}}
    \newcommand{\FunctionTok}[1]{\textcolor[rgb]{0.02,0.16,0.49}{{#1}}}
    \newcommand{\RegionMarkerTok}[1]{{#1}}
    \newcommand{\ErrorTok}[1]{\textcolor[rgb]{1.00,0.00,0.00}{\textbf{{#1}}}}
    \newcommand{\NormalTok}[1]{{#1}}
    
    % Additional commands for more recent versions of Pandoc
    \newcommand{\ConstantTok}[1]{\textcolor[rgb]{0.53,0.00,0.00}{{#1}}}
    \newcommand{\SpecialCharTok}[1]{\textcolor[rgb]{0.25,0.44,0.63}{{#1}}}
    \newcommand{\VerbatimStringTok}[1]{\textcolor[rgb]{0.25,0.44,0.63}{{#1}}}
    \newcommand{\SpecialStringTok}[1]{\textcolor[rgb]{0.73,0.40,0.53}{{#1}}}
    \newcommand{\ImportTok}[1]{{#1}}
    \newcommand{\DocumentationTok}[1]{\textcolor[rgb]{0.73,0.13,0.13}{\textit{{#1}}}}
    \newcommand{\AnnotationTok}[1]{\textcolor[rgb]{0.38,0.63,0.69}{\textbf{\textit{{#1}}}}}
    \newcommand{\CommentVarTok}[1]{\textcolor[rgb]{0.38,0.63,0.69}{\textbf{\textit{{#1}}}}}
    \newcommand{\VariableTok}[1]{\textcolor[rgb]{0.10,0.09,0.49}{{#1}}}
    \newcommand{\ControlFlowTok}[1]{\textcolor[rgb]{0.00,0.44,0.13}{\textbf{{#1}}}}
    \newcommand{\OperatorTok}[1]{\textcolor[rgb]{0.40,0.40,0.40}{{#1}}}
    \newcommand{\BuiltInTok}[1]{{#1}}
    \newcommand{\ExtensionTok}[1]{{#1}}
    \newcommand{\PreprocessorTok}[1]{\textcolor[rgb]{0.74,0.48,0.00}{{#1}}}
    \newcommand{\AttributeTok}[1]{\textcolor[rgb]{0.49,0.56,0.16}{{#1}}}
    \newcommand{\InformationTok}[1]{\textcolor[rgb]{0.38,0.63,0.69}{\textbf{\textit{{#1}}}}}
    \newcommand{\WarningTok}[1]{\textcolor[rgb]{0.38,0.63,0.69}{\textbf{\textit{{#1}}}}}
    
    
    % Define a nice break command that doesn't care if a line doesn't already
    % exist.
    \def\br{\hspace*{\fill} \\* }
    % Math Jax compatibility definitions
    \def\gt{>}
    \def\lt{<}
    \let\Oldtex\TeX
    \let\Oldlatex\LaTeX
    \renewcommand{\TeX}{\textrm{\Oldtex}}
    \renewcommand{\LaTeX}{\textrm{\Oldlatex}}
    % Document parameters
    % Document title
    \title{Ford\_Go\_Bike\_Part1}
    
    
    
    
    
% Pygments definitions
\makeatletter
\def\PY@reset{\let\PY@it=\relax \let\PY@bf=\relax%
    \let\PY@ul=\relax \let\PY@tc=\relax%
    \let\PY@bc=\relax \let\PY@ff=\relax}
\def\PY@tok#1{\csname PY@tok@#1\endcsname}
\def\PY@toks#1+{\ifx\relax#1\empty\else%
    \PY@tok{#1}\expandafter\PY@toks\fi}
\def\PY@do#1{\PY@bc{\PY@tc{\PY@ul{%
    \PY@it{\PY@bf{\PY@ff{#1}}}}}}}
\def\PY#1#2{\PY@reset\PY@toks#1+\relax+\PY@do{#2}}

\@namedef{PY@tok@w}{\def\PY@tc##1{\textcolor[rgb]{0.73,0.73,0.73}{##1}}}
\@namedef{PY@tok@c}{\let\PY@it=\textit\def\PY@tc##1{\textcolor[rgb]{0.24,0.48,0.48}{##1}}}
\@namedef{PY@tok@cp}{\def\PY@tc##1{\textcolor[rgb]{0.61,0.40,0.00}{##1}}}
\@namedef{PY@tok@k}{\let\PY@bf=\textbf\def\PY@tc##1{\textcolor[rgb]{0.00,0.50,0.00}{##1}}}
\@namedef{PY@tok@kp}{\def\PY@tc##1{\textcolor[rgb]{0.00,0.50,0.00}{##1}}}
\@namedef{PY@tok@kt}{\def\PY@tc##1{\textcolor[rgb]{0.69,0.00,0.25}{##1}}}
\@namedef{PY@tok@o}{\def\PY@tc##1{\textcolor[rgb]{0.40,0.40,0.40}{##1}}}
\@namedef{PY@tok@ow}{\let\PY@bf=\textbf\def\PY@tc##1{\textcolor[rgb]{0.67,0.13,1.00}{##1}}}
\@namedef{PY@tok@nb}{\def\PY@tc##1{\textcolor[rgb]{0.00,0.50,0.00}{##1}}}
\@namedef{PY@tok@nf}{\def\PY@tc##1{\textcolor[rgb]{0.00,0.00,1.00}{##1}}}
\@namedef{PY@tok@nc}{\let\PY@bf=\textbf\def\PY@tc##1{\textcolor[rgb]{0.00,0.00,1.00}{##1}}}
\@namedef{PY@tok@nn}{\let\PY@bf=\textbf\def\PY@tc##1{\textcolor[rgb]{0.00,0.00,1.00}{##1}}}
\@namedef{PY@tok@ne}{\let\PY@bf=\textbf\def\PY@tc##1{\textcolor[rgb]{0.80,0.25,0.22}{##1}}}
\@namedef{PY@tok@nv}{\def\PY@tc##1{\textcolor[rgb]{0.10,0.09,0.49}{##1}}}
\@namedef{PY@tok@no}{\def\PY@tc##1{\textcolor[rgb]{0.53,0.00,0.00}{##1}}}
\@namedef{PY@tok@nl}{\def\PY@tc##1{\textcolor[rgb]{0.46,0.46,0.00}{##1}}}
\@namedef{PY@tok@ni}{\let\PY@bf=\textbf\def\PY@tc##1{\textcolor[rgb]{0.44,0.44,0.44}{##1}}}
\@namedef{PY@tok@na}{\def\PY@tc##1{\textcolor[rgb]{0.41,0.47,0.13}{##1}}}
\@namedef{PY@tok@nt}{\let\PY@bf=\textbf\def\PY@tc##1{\textcolor[rgb]{0.00,0.50,0.00}{##1}}}
\@namedef{PY@tok@nd}{\def\PY@tc##1{\textcolor[rgb]{0.67,0.13,1.00}{##1}}}
\@namedef{PY@tok@s}{\def\PY@tc##1{\textcolor[rgb]{0.73,0.13,0.13}{##1}}}
\@namedef{PY@tok@sd}{\let\PY@it=\textit\def\PY@tc##1{\textcolor[rgb]{0.73,0.13,0.13}{##1}}}
\@namedef{PY@tok@si}{\let\PY@bf=\textbf\def\PY@tc##1{\textcolor[rgb]{0.64,0.35,0.47}{##1}}}
\@namedef{PY@tok@se}{\let\PY@bf=\textbf\def\PY@tc##1{\textcolor[rgb]{0.67,0.36,0.12}{##1}}}
\@namedef{PY@tok@sr}{\def\PY@tc##1{\textcolor[rgb]{0.64,0.35,0.47}{##1}}}
\@namedef{PY@tok@ss}{\def\PY@tc##1{\textcolor[rgb]{0.10,0.09,0.49}{##1}}}
\@namedef{PY@tok@sx}{\def\PY@tc##1{\textcolor[rgb]{0.00,0.50,0.00}{##1}}}
\@namedef{PY@tok@m}{\def\PY@tc##1{\textcolor[rgb]{0.40,0.40,0.40}{##1}}}
\@namedef{PY@tok@gh}{\let\PY@bf=\textbf\def\PY@tc##1{\textcolor[rgb]{0.00,0.00,0.50}{##1}}}
\@namedef{PY@tok@gu}{\let\PY@bf=\textbf\def\PY@tc##1{\textcolor[rgb]{0.50,0.00,0.50}{##1}}}
\@namedef{PY@tok@gd}{\def\PY@tc##1{\textcolor[rgb]{0.63,0.00,0.00}{##1}}}
\@namedef{PY@tok@gi}{\def\PY@tc##1{\textcolor[rgb]{0.00,0.52,0.00}{##1}}}
\@namedef{PY@tok@gr}{\def\PY@tc##1{\textcolor[rgb]{0.89,0.00,0.00}{##1}}}
\@namedef{PY@tok@ge}{\let\PY@it=\textit}
\@namedef{PY@tok@gs}{\let\PY@bf=\textbf}
\@namedef{PY@tok@gp}{\let\PY@bf=\textbf\def\PY@tc##1{\textcolor[rgb]{0.00,0.00,0.50}{##1}}}
\@namedef{PY@tok@go}{\def\PY@tc##1{\textcolor[rgb]{0.44,0.44,0.44}{##1}}}
\@namedef{PY@tok@gt}{\def\PY@tc##1{\textcolor[rgb]{0.00,0.27,0.87}{##1}}}
\@namedef{PY@tok@err}{\def\PY@bc##1{{\setlength{\fboxsep}{\string -\fboxrule}\fcolorbox[rgb]{1.00,0.00,0.00}{1,1,1}{\strut ##1}}}}
\@namedef{PY@tok@kc}{\let\PY@bf=\textbf\def\PY@tc##1{\textcolor[rgb]{0.00,0.50,0.00}{##1}}}
\@namedef{PY@tok@kd}{\let\PY@bf=\textbf\def\PY@tc##1{\textcolor[rgb]{0.00,0.50,0.00}{##1}}}
\@namedef{PY@tok@kn}{\let\PY@bf=\textbf\def\PY@tc##1{\textcolor[rgb]{0.00,0.50,0.00}{##1}}}
\@namedef{PY@tok@kr}{\let\PY@bf=\textbf\def\PY@tc##1{\textcolor[rgb]{0.00,0.50,0.00}{##1}}}
\@namedef{PY@tok@bp}{\def\PY@tc##1{\textcolor[rgb]{0.00,0.50,0.00}{##1}}}
\@namedef{PY@tok@fm}{\def\PY@tc##1{\textcolor[rgb]{0.00,0.00,1.00}{##1}}}
\@namedef{PY@tok@vc}{\def\PY@tc##1{\textcolor[rgb]{0.10,0.09,0.49}{##1}}}
\@namedef{PY@tok@vg}{\def\PY@tc##1{\textcolor[rgb]{0.10,0.09,0.49}{##1}}}
\@namedef{PY@tok@vi}{\def\PY@tc##1{\textcolor[rgb]{0.10,0.09,0.49}{##1}}}
\@namedef{PY@tok@vm}{\def\PY@tc##1{\textcolor[rgb]{0.10,0.09,0.49}{##1}}}
\@namedef{PY@tok@sa}{\def\PY@tc##1{\textcolor[rgb]{0.73,0.13,0.13}{##1}}}
\@namedef{PY@tok@sb}{\def\PY@tc##1{\textcolor[rgb]{0.73,0.13,0.13}{##1}}}
\@namedef{PY@tok@sc}{\def\PY@tc##1{\textcolor[rgb]{0.73,0.13,0.13}{##1}}}
\@namedef{PY@tok@dl}{\def\PY@tc##1{\textcolor[rgb]{0.73,0.13,0.13}{##1}}}
\@namedef{PY@tok@s2}{\def\PY@tc##1{\textcolor[rgb]{0.73,0.13,0.13}{##1}}}
\@namedef{PY@tok@sh}{\def\PY@tc##1{\textcolor[rgb]{0.73,0.13,0.13}{##1}}}
\@namedef{PY@tok@s1}{\def\PY@tc##1{\textcolor[rgb]{0.73,0.13,0.13}{##1}}}
\@namedef{PY@tok@mb}{\def\PY@tc##1{\textcolor[rgb]{0.40,0.40,0.40}{##1}}}
\@namedef{PY@tok@mf}{\def\PY@tc##1{\textcolor[rgb]{0.40,0.40,0.40}{##1}}}
\@namedef{PY@tok@mh}{\def\PY@tc##1{\textcolor[rgb]{0.40,0.40,0.40}{##1}}}
\@namedef{PY@tok@mi}{\def\PY@tc##1{\textcolor[rgb]{0.40,0.40,0.40}{##1}}}
\@namedef{PY@tok@il}{\def\PY@tc##1{\textcolor[rgb]{0.40,0.40,0.40}{##1}}}
\@namedef{PY@tok@mo}{\def\PY@tc##1{\textcolor[rgb]{0.40,0.40,0.40}{##1}}}
\@namedef{PY@tok@ch}{\let\PY@it=\textit\def\PY@tc##1{\textcolor[rgb]{0.24,0.48,0.48}{##1}}}
\@namedef{PY@tok@cm}{\let\PY@it=\textit\def\PY@tc##1{\textcolor[rgb]{0.24,0.48,0.48}{##1}}}
\@namedef{PY@tok@cpf}{\let\PY@it=\textit\def\PY@tc##1{\textcolor[rgb]{0.24,0.48,0.48}{##1}}}
\@namedef{PY@tok@c1}{\let\PY@it=\textit\def\PY@tc##1{\textcolor[rgb]{0.24,0.48,0.48}{##1}}}
\@namedef{PY@tok@cs}{\let\PY@it=\textit\def\PY@tc##1{\textcolor[rgb]{0.24,0.48,0.48}{##1}}}

\def\PYZbs{\char`\\}
\def\PYZus{\char`\_}
\def\PYZob{\char`\{}
\def\PYZcb{\char`\}}
\def\PYZca{\char`\^}
\def\PYZam{\char`\&}
\def\PYZlt{\char`\<}
\def\PYZgt{\char`\>}
\def\PYZsh{\char`\#}
\def\PYZpc{\char`\%}
\def\PYZdl{\char`\$}
\def\PYZhy{\char`\-}
\def\PYZsq{\char`\'}
\def\PYZdq{\char`\"}
\def\PYZti{\char`\~}
% for compatibility with earlier versions
\def\PYZat{@}
\def\PYZlb{[}
\def\PYZrb{]}
\makeatother


    % For linebreaks inside Verbatim environment from package fancyvrb. 
    \makeatletter
        \newbox\Wrappedcontinuationbox 
        \newbox\Wrappedvisiblespacebox 
        \newcommand*\Wrappedvisiblespace {\textcolor{red}{\textvisiblespace}} 
        \newcommand*\Wrappedcontinuationsymbol {\textcolor{red}{\llap{\tiny$\m@th\hookrightarrow$}}} 
        \newcommand*\Wrappedcontinuationindent {3ex } 
        \newcommand*\Wrappedafterbreak {\kern\Wrappedcontinuationindent\copy\Wrappedcontinuationbox} 
        % Take advantage of the already applied Pygments mark-up to insert 
        % potential linebreaks for TeX processing. 
        %        {, <, #, %, $, ' and ": go to next line. 
        %        _, }, ^, &, >, - and ~: stay at end of broken line. 
        % Use of \textquotesingle for straight quote. 
        \newcommand*\Wrappedbreaksatspecials {% 
            \def\PYGZus{\discretionary{\char`\_}{\Wrappedafterbreak}{\char`\_}}% 
            \def\PYGZob{\discretionary{}{\Wrappedafterbreak\char`\{}{\char`\{}}% 
            \def\PYGZcb{\discretionary{\char`\}}{\Wrappedafterbreak}{\char`\}}}% 
            \def\PYGZca{\discretionary{\char`\^}{\Wrappedafterbreak}{\char`\^}}% 
            \def\PYGZam{\discretionary{\char`\&}{\Wrappedafterbreak}{\char`\&}}% 
            \def\PYGZlt{\discretionary{}{\Wrappedafterbreak\char`\<}{\char`\<}}% 
            \def\PYGZgt{\discretionary{\char`\>}{\Wrappedafterbreak}{\char`\>}}% 
            \def\PYGZsh{\discretionary{}{\Wrappedafterbreak\char`\#}{\char`\#}}% 
            \def\PYGZpc{\discretionary{}{\Wrappedafterbreak\char`\%}{\char`\%}}% 
            \def\PYGZdl{\discretionary{}{\Wrappedafterbreak\char`\$}{\char`\$}}% 
            \def\PYGZhy{\discretionary{\char`\-}{\Wrappedafterbreak}{\char`\-}}% 
            \def\PYGZsq{\discretionary{}{\Wrappedafterbreak\textquotesingle}{\textquotesingle}}% 
            \def\PYGZdq{\discretionary{}{\Wrappedafterbreak\char`\"}{\char`\"}}% 
            \def\PYGZti{\discretionary{\char`\~}{\Wrappedafterbreak}{\char`\~}}% 
        } 
        % Some characters . , ; ? ! / are not pygmentized. 
        % This macro makes them "active" and they will insert potential linebreaks 
        \newcommand*\Wrappedbreaksatpunct {% 
            \lccode`\~`\.\lowercase{\def~}{\discretionary{\hbox{\char`\.}}{\Wrappedafterbreak}{\hbox{\char`\.}}}% 
            \lccode`\~`\,\lowercase{\def~}{\discretionary{\hbox{\char`\,}}{\Wrappedafterbreak}{\hbox{\char`\,}}}% 
            \lccode`\~`\;\lowercase{\def~}{\discretionary{\hbox{\char`\;}}{\Wrappedafterbreak}{\hbox{\char`\;}}}% 
            \lccode`\~`\:\lowercase{\def~}{\discretionary{\hbox{\char`\:}}{\Wrappedafterbreak}{\hbox{\char`\:}}}% 
            \lccode`\~`\?\lowercase{\def~}{\discretionary{\hbox{\char`\?}}{\Wrappedafterbreak}{\hbox{\char`\?}}}% 
            \lccode`\~`\!\lowercase{\def~}{\discretionary{\hbox{\char`\!}}{\Wrappedafterbreak}{\hbox{\char`\!}}}% 
            \lccode`\~`\/\lowercase{\def~}{\discretionary{\hbox{\char`\/}}{\Wrappedafterbreak}{\hbox{\char`\/}}}% 
            \catcode`\.\active
            \catcode`\,\active 
            \catcode`\;\active
            \catcode`\:\active
            \catcode`\?\active
            \catcode`\!\active
            \catcode`\/\active 
            \lccode`\~`\~ 	
        }
    \makeatother

    \let\OriginalVerbatim=\Verbatim
    \makeatletter
    \renewcommand{\Verbatim}[1][1]{%
        %\parskip\z@skip
        \sbox\Wrappedcontinuationbox {\Wrappedcontinuationsymbol}%
        \sbox\Wrappedvisiblespacebox {\FV@SetupFont\Wrappedvisiblespace}%
        \def\FancyVerbFormatLine ##1{\hsize\linewidth
            \vtop{\raggedright\hyphenpenalty\z@\exhyphenpenalty\z@
                \doublehyphendemerits\z@\finalhyphendemerits\z@
                \strut ##1\strut}%
        }%
        % If the linebreak is at a space, the latter will be displayed as visible
        % space at end of first line, and a continuation symbol starts next line.
        % Stretch/shrink are however usually zero for typewriter font.
        \def\FV@Space {%
            \nobreak\hskip\z@ plus\fontdimen3\font minus\fontdimen4\font
            \discretionary{\copy\Wrappedvisiblespacebox}{\Wrappedafterbreak}
            {\kern\fontdimen2\font}%
        }%
        
        % Allow breaks at special characters using \PYG... macros.
        \Wrappedbreaksatspecials
        % Breaks at punctuation characters . , ; ? ! and / need catcode=\active 	
        \OriginalVerbatim[#1,codes*=\Wrappedbreaksatpunct]%
    }
    \makeatother

    % Exact colors from NB
    \definecolor{incolor}{HTML}{303F9F}
    \definecolor{outcolor}{HTML}{D84315}
    \definecolor{cellborder}{HTML}{CFCFCF}
    \definecolor{cellbackground}{HTML}{F7F7F7}
    
    % prompt
    \makeatletter
    \newcommand{\boxspacing}{\kern\kvtcb@left@rule\kern\kvtcb@boxsep}
    \makeatother
    \newcommand{\prompt}[4]{
        {\ttfamily\llap{{\color{#2}[#3]:\hspace{3pt}#4}}\vspace{-\baselineskip}}
    }
    

    
    % Prevent overflowing lines due to hard-to-break entities
    \sloppy 
    % Setup hyperref package
    \hypersetup{
      breaklinks=true,  % so long urls are correctly broken across lines
      colorlinks=true,
      urlcolor=urlcolor,
      linkcolor=linkcolor,
      citecolor=citecolor,
      }
    % Slightly bigger margins than the latex defaults
    
    \geometry{verbose,tmargin=1in,bmargin=1in,lmargin=1in,rmargin=1in}
    
    

\begin{document}
    
    \maketitle
    
    

    
    \hypertarget{project-ford-gobike}{%
\section{Project : Ford GoBike}\label{project-ford-gobike}}

\hypertarget{table-of-contents}{%
\subsection{Table of Contents}\label{table-of-contents}}

Introduction

Data Wrangling

Exploratory Data Analysis

    \#\# Introduction

\hypertarget{dataset-description}{%
\subsubsection{Dataset Description}\label{dataset-description}}

\hypertarget{what-features-in-the-dataset-do-you-think-will-help-support-your-investigation-into-your-features-of-interest}{%
\subsubsection{What features in the dataset do you think will help
support your investigation into your feature(s) of
interest?}\label{what-features-in-the-dataset-do-you-think-will-help-support-your-investigation-into-your-features-of-interest}}

\begin{quote}
\textbf{Tip}: information about individual rides made in a bike-sharing
system covering the greater San Francisco Bay area. ● Note that this
dataset will require some data wrangling in order to make it tidy for
analysis. There are multiple cities covered by the linked system, and
multiple data files will need to be joined together if a full year's
coverage is desired. Here are the datasets in CSV format. You can fit
your model using the train data, then predict using the test data and
submit your predictions in the format of the sample submission. Your
goal is to predict the rotor bearing temperature, which is the Target
column in the datasets.
\href{https://www.google.com/url?q=https://video.udacity-data.com/topher/2020/October/5f91cf38_201902-fordgobike-tripdata/201902-fordgobike-tripdata.csv\&sa=D\&source=editors\&ust=1669750197727856\&usg=AOvVaw0RJVqpWyfu7RoKaPH1gynL}{here}.
Files
\end{quote}

\hypertarget{what-is-the-structure-of-your-dataset}{%
\subsubsection{What is the structure of your
dataset?}\label{what-is-the-structure-of-your-dataset}}

\hypertarget{data-dictionary}{%
\paragraph{Data Dictionary}\label{data-dictionary}}

01 - duration\_sec

02 - start\_time

03 - end\_time

04 - start\_station\_id

05 - start\_station\_name

06 - start\_station\_latitude

07 - start\_station\_longitude

08 - end\_station\_id

09 - end\_station\_name

10 - end\_station\_latitude

11 - end\_station\_longitude

12 - bike\_id

13 - user\_type

14 - member\_birth\_year

15 - member\_gender

16 - bike\_share\_for\_all\_trip

\hypertarget{what-isare-the-main-features-of-interest-in-your-dataset}{%
\subsubsection{What is/are the main feature(s) of interest in your
dataset?}\label{what-isare-the-main-features-of-interest-in-your-dataset}}

I'm most interested in figuring out what features are best for
predicting most trips taken in terms of time of day, day of the week, or
month of the year? and How long does the average trip take? Does the
above depend on if a user is a subscriber or customer?.

    \begin{tcolorbox}[breakable, size=fbox, boxrule=1pt, pad at break*=1mm,colback=cellbackground, colframe=cellborder]
\prompt{In}{incolor}{1}{\boxspacing}
\begin{Verbatim}[commandchars=\\\{\}]
\PY{k+kn}{import} \PY{n+nn}{numpy} \PY{k}{as} \PY{n+nn}{np}
\PY{k+kn}{import} \PY{n+nn}{pandas} \PY{k}{as} \PY{n+nn}{pd} 
\PY{k+kn}{import} \PY{n+nn}{seaborn} \PY{k}{as} \PY{n+nn}{sns}
\PY{k+kn}{import} \PY{n+nn}{matplotlib}\PY{n+nn}{.}\PY{n+nn}{pyplot} \PY{k}{as} \PY{n+nn}{plt}
\PY{k+kn}{from} \PY{n+nn}{matplotlib} \PY{k+kn}{import} \PY{n}{colors}
\PY{k+kn}{from} \PY{n+nn}{matplotlib}\PY{n+nn}{.}\PY{n+nn}{ticker} \PY{k+kn}{import} \PY{n}{PercentFormatter}

\PY{k+kn}{from} \PY{n+nn}{pathlib} \PY{k+kn}{import} \PY{n}{Path}
\PY{k+kn}{from} \PY{n+nn}{warnings} \PY{k+kn}{import} \PY{n}{simplefilter}

\PY{k+kn}{import} \PY{n+nn}{requests}

\PY{n}{simplefilter}\PY{p}{(}\PY{l+s+s2}{\PYZdq{}}\PY{l+s+s2}{ignore}\PY{l+s+s2}{\PYZdq{}}\PY{p}{)}

\PY{o}{\PYZpc{}}\PY{k}{matplotlib} inline
\end{Verbatim}
\end{tcolorbox}

    \#\# Data Wrangling

\begin{quote}
\textbf{Tip}: In this section of the report, you will load in the data,
check for cleanliness, and then trim and clean your dataset for
analysis. Make sure that you \textbf{document your data cleaning steps
in mark-down cells precisely and justify your cleaning decisions.}
\end{quote}

    Data Wrangling which include : 1.Gathering Data 2.Assessing Data
3.cleaning Data

    \hypertarget{gathering-data}{%
\subsubsection{Gathering Data}\label{gathering-data}}

    \begin{tcolorbox}[breakable, size=fbox, boxrule=1pt, pad at break*=1mm,colback=cellbackground, colframe=cellborder]
\prompt{In}{incolor}{2}{\boxspacing}
\begin{Verbatim}[commandchars=\\\{\}]
\PY{c+c1}{\PYZsh{} Load your data and print out a few lines. Perform operations to inspect data}
\PY{c+c1}{\PYZsh{}   types and look for instances of missing or possibly errant data.}
\PY{n}{df} \PY{o}{=} \PY{n}{pd}\PY{o}{.}\PY{n}{read\PYZus{}csv}\PY{p}{(}\PY{l+s+s1}{\PYZsq{}}\PY{l+s+s1}{data/201902\PYZhy{}fordgobike\PYZhy{}tripdata.csv}\PY{l+s+s1}{\PYZsq{}}\PY{p}{)}
\end{Verbatim}
\end{tcolorbox}

    we aqucistion data from dataset like : csv file in our example

    \begin{tcolorbox}[breakable, size=fbox, boxrule=1pt, pad at break*=1mm,colback=cellbackground, colframe=cellborder]
\prompt{In}{incolor}{3}{\boxspacing}
\begin{Verbatim}[commandchars=\\\{\}]
\PY{n}{df}\PY{o}{.}\PY{n}{head}\PY{p}{(}\PY{p}{)}
\end{Verbatim}
\end{tcolorbox}

            \begin{tcolorbox}[breakable, size=fbox, boxrule=.5pt, pad at break*=1mm, opacityfill=0]
\prompt{Out}{outcolor}{3}{\boxspacing}
\begin{Verbatim}[commandchars=\\\{\}]
   duration\_sec                start\_time                  end\_time  \textbackslash{}
0         52185  2019-02-28 17:32:10.1450  2019-03-01 08:01:55.9750
1         42521  2019-02-28 18:53:21.7890  2019-03-01 06:42:03.0560
2         61854  2019-02-28 12:13:13.2180  2019-03-01 05:24:08.1460
3         36490  2019-02-28 17:54:26.0100  2019-03-01 04:02:36.8420
4          1585  2019-02-28 23:54:18.5490  2019-03-01 00:20:44.0740

   start\_station\_id                                start\_station\_name  \textbackslash{}
0              21.0  Montgomery St BART Station (Market St at 2nd St)
1              23.0                     The Embarcadero at Steuart St
2              86.0                           Market St at Dolores St
3             375.0                           Grove St at Masonic Ave
4               7.0                               Frank H Ogawa Plaza

   start\_station\_latitude  start\_station\_longitude  end\_station\_id  \textbackslash{}
0               37.789625              -122.400811            13.0
1               37.791464              -122.391034            81.0
2               37.769305              -122.426826             3.0
3               37.774836              -122.446546            70.0
4               37.804562              -122.271738           222.0

                               end\_station\_name  end\_station\_latitude  \textbackslash{}
0                Commercial St at Montgomery St             37.794231
1                            Berry St at 4th St             37.775880
2  Powell St BART Station (Market St at 4th St)             37.786375
3                        Central Ave at Fell St             37.773311
4                         10th Ave at E 15th St             37.792714

   end\_station\_longitude  bike\_id   user\_type  member\_birth\_year  \textbackslash{}
0            -122.402923     4902    Customer             1984.0
1            -122.393170     2535    Customer                NaN
2            -122.404904     5905    Customer             1972.0
3            -122.444293     6638  Subscriber             1989.0
4            -122.248780     4898  Subscriber             1974.0

  member\_gender bike\_share\_for\_all\_trip
0          Male                      No
1           NaN                      No
2          Male                      No
3         Other                      No
4          Male                     Yes
\end{Verbatim}
\end{tcolorbox}
        
    we use head() or tail() function to display a sample of data

    \begin{tcolorbox}[breakable, size=fbox, boxrule=1pt, pad at break*=1mm,colback=cellbackground, colframe=cellborder]
\prompt{In}{incolor}{4}{\boxspacing}
\begin{Verbatim}[commandchars=\\\{\}]
\PY{n}{df}\PY{o}{.}\PY{n}{tail}\PY{p}{(}\PY{p}{)}
\end{Verbatim}
\end{tcolorbox}

            \begin{tcolorbox}[breakable, size=fbox, boxrule=.5pt, pad at break*=1mm, opacityfill=0]
\prompt{Out}{outcolor}{4}{\boxspacing}
\begin{Verbatim}[commandchars=\\\{\}]
        duration\_sec                start\_time                  end\_time  \textbackslash{}
183407           480  2019-02-01 00:04:49.7240  2019-02-01 00:12:50.0340
183408           313  2019-02-01 00:05:34.7440  2019-02-01 00:10:48.5020
183409           141  2019-02-01 00:06:05.5490  2019-02-01 00:08:27.2200
183410           139  2019-02-01 00:05:34.3600  2019-02-01 00:07:54.2870
183411           271  2019-02-01 00:00:20.6360  2019-02-01 00:04:52.0580

        start\_station\_id                                start\_station\_name  \textbackslash{}
183407              27.0                           Beale St at Harrison St
183408              21.0  Montgomery St BART Station (Market St at 2nd St)
183409             278.0                            The Alameda at Bush St
183410             220.0                       San Pablo Ave at MLK Jr Way
183411              24.0                             Spear St at Folsom St

        start\_station\_latitude  start\_station\_longitude  end\_station\_id  \textbackslash{}
183407               37.788059              -122.391865           324.0
183408               37.789625              -122.400811            66.0
183409               37.331932              -121.904888           277.0
183410               37.811351              -122.273422           216.0
183411               37.789677              -122.390428            37.0

                           end\_station\_name  end\_station\_latitude  \textbackslash{}
183407  Union Square (Powell St at Post St)             37.788300
183408                3rd St at Townsend St             37.778742
183409            Morrison Ave at Julian St             37.333658
183410             San Pablo Ave at 27th St             37.817827
183411                  2nd St at Folsom St             37.785000

        end\_station\_longitude  bike\_id   user\_type  member\_birth\_year  \textbackslash{}
183407            -122.408531     4832  Subscriber             1996.0
183408            -122.392741     4960  Subscriber             1984.0
183409            -121.908586     3824  Subscriber             1990.0
183410            -122.275698     5095  Subscriber             1988.0
183411            -122.395936     1057  Subscriber             1989.0

       member\_gender bike\_share\_for\_all\_trip
183407          Male                      No
183408          Male                      No
183409          Male                     Yes
183410          Male                      No
183411          Male                      No
\end{Verbatim}
\end{tcolorbox}
        
    \hypertarget{assessing-data}{%
\subsubsection{Assessing Data}\label{assessing-data}}

    We assessing our data using some function like : shape , ndim , dtypes ,
size , info() , nunique() , isnull()

    \begin{tcolorbox}[breakable, size=fbox, boxrule=1pt, pad at break*=1mm,colback=cellbackground, colframe=cellborder]
\prompt{In}{incolor}{5}{\boxspacing}
\begin{Verbatim}[commandchars=\\\{\}]
\PY{c+c1}{\PYZsh{} return number of columns and number of row}
\PY{n}{df}\PY{o}{.}\PY{n}{shape}
\end{Verbatim}
\end{tcolorbox}

            \begin{tcolorbox}[breakable, size=fbox, boxrule=.5pt, pad at break*=1mm, opacityfill=0]
\prompt{Out}{outcolor}{5}{\boxspacing}
\begin{Verbatim}[commandchars=\\\{\}]
(183412, 16)
\end{Verbatim}
\end{tcolorbox}
        
    the shape function get number of rows and number of columns in tuple

    \begin{tcolorbox}[breakable, size=fbox, boxrule=1pt, pad at break*=1mm,colback=cellbackground, colframe=cellborder]
\prompt{In}{incolor}{6}{\boxspacing}
\begin{Verbatim}[commandchars=\\\{\}]
\PY{c+c1}{\PYZsh{}return number of dimensions of data}
\PY{n}{df}\PY{o}{.}\PY{n}{ndim}
\end{Verbatim}
\end{tcolorbox}

            \begin{tcolorbox}[breakable, size=fbox, boxrule=.5pt, pad at break*=1mm, opacityfill=0]
\prompt{Out}{outcolor}{6}{\boxspacing}
\begin{Verbatim}[commandchars=\\\{\}]
2
\end{Verbatim}
\end{tcolorbox}
        
    size function show us the result of multiplication of number of rows and
number of columns

    \begin{tcolorbox}[breakable, size=fbox, boxrule=1pt, pad at break*=1mm,colback=cellbackground, colframe=cellborder]
\prompt{In}{incolor}{7}{\boxspacing}
\begin{Verbatim}[commandchars=\\\{\}]
\PY{c+c1}{\PYZsh{} return size of Dataset which is a multiplication of number of rows and number of columns}
\PY{n}{df}\PY{o}{.}\PY{n}{size}
\end{Verbatim}
\end{tcolorbox}

            \begin{tcolorbox}[breakable, size=fbox, boxrule=.5pt, pad at break*=1mm, opacityfill=0]
\prompt{Out}{outcolor}{7}{\boxspacing}
\begin{Verbatim}[commandchars=\\\{\}]
2934592
\end{Verbatim}
\end{tcolorbox}
        
    dtypes show us data type of each column (features)

    \begin{tcolorbox}[breakable, size=fbox, boxrule=1pt, pad at break*=1mm,colback=cellbackground, colframe=cellborder]
\prompt{In}{incolor}{8}{\boxspacing}
\begin{Verbatim}[commandchars=\\\{\}]
\PY{c+c1}{\PYZsh{}return types of each column}
\PY{n}{df}\PY{o}{.}\PY{n}{dtypes}
\end{Verbatim}
\end{tcolorbox}

            \begin{tcolorbox}[breakable, size=fbox, boxrule=.5pt, pad at break*=1mm, opacityfill=0]
\prompt{Out}{outcolor}{8}{\boxspacing}
\begin{Verbatim}[commandchars=\\\{\}]
duration\_sec                 int64
start\_time                  object
end\_time                    object
start\_station\_id           float64
start\_station\_name          object
start\_station\_latitude     float64
start\_station\_longitude    float64
end\_station\_id             float64
end\_station\_name            object
end\_station\_latitude       float64
end\_station\_longitude      float64
bike\_id                      int64
user\_type                   object
member\_birth\_year          float64
member\_gender               object
bike\_share\_for\_all\_trip     object
dtype: object
\end{Verbatim}
\end{tcolorbox}
        
    info() function show us number of non\_null\_value in each column and
datatype

it has two features(no of non\_null\_value,datatype)

    \begin{tcolorbox}[breakable, size=fbox, boxrule=1pt, pad at break*=1mm,colback=cellbackground, colframe=cellborder]
\prompt{In}{incolor}{9}{\boxspacing}
\begin{Verbatim}[commandchars=\\\{\}]
\PY{c+c1}{\PYZsh{}return number of non\PYZhy{}null\PYZhy{}value and datatype of each column}
\PY{n}{df}\PY{o}{.}\PY{n}{info}\PY{p}{(}\PY{p}{)}
\end{Verbatim}
\end{tcolorbox}

    \begin{Verbatim}[commandchars=\\\{\}]
<class 'pandas.core.frame.DataFrame'>
RangeIndex: 183412 entries, 0 to 183411
Data columns (total 16 columns):
 \#   Column                   Non-Null Count   Dtype
---  ------                   --------------   -----
 0   duration\_sec             183412 non-null  int64
 1   start\_time               183412 non-null  object
 2   end\_time                 183412 non-null  object
 3   start\_station\_id         183215 non-null  float64
 4   start\_station\_name       183215 non-null  object
 5   start\_station\_latitude   183412 non-null  float64
 6   start\_station\_longitude  183412 non-null  float64
 7   end\_station\_id           183215 non-null  float64
 8   end\_station\_name         183215 non-null  object
 9   end\_station\_latitude     183412 non-null  float64
 10  end\_station\_longitude    183412 non-null  float64
 11  bike\_id                  183412 non-null  int64
 12  user\_type                183412 non-null  object
 13  member\_birth\_year        175147 non-null  float64
 14  member\_gender            175147 non-null  object
 15  bike\_share\_for\_all\_trip  183412 non-null  object
dtypes: float64(7), int64(2), object(7)
memory usage: 22.4+ MB
    \end{Verbatim}

    nunique() show us number of unique values in each column

    \begin{tcolorbox}[breakable, size=fbox, boxrule=1pt, pad at break*=1mm,colback=cellbackground, colframe=cellborder]
\prompt{In}{incolor}{10}{\boxspacing}
\begin{Verbatim}[commandchars=\\\{\}]
\PY{c+c1}{\PYZsh{}return number of unique value}
\PY{n}{df}\PY{o}{.}\PY{n}{nunique}\PY{p}{(}\PY{p}{)}
\end{Verbatim}
\end{tcolorbox}

            \begin{tcolorbox}[breakable, size=fbox, boxrule=.5pt, pad at break*=1mm, opacityfill=0]
\prompt{Out}{outcolor}{10}{\boxspacing}
\begin{Verbatim}[commandchars=\\\{\}]
duration\_sec                 4752
start\_time                 183401
end\_time                   183397
start\_station\_id              329
start\_station\_name            329
start\_station\_latitude        334
start\_station\_longitude       335
end\_station\_id                329
end\_station\_name              329
end\_station\_latitude          335
end\_station\_longitude         335
bike\_id                      4646
user\_type                       2
member\_birth\_year              75
member\_gender                   3
bike\_share\_for\_all\_trip         2
dtype: int64
\end{Verbatim}
\end{tcolorbox}
        
    isnull() function show us boolean value for each element (each cell) it
is null or not

if it(element) null return True

else return False

    \begin{tcolorbox}[breakable, size=fbox, boxrule=1pt, pad at break*=1mm,colback=cellbackground, colframe=cellborder]
\prompt{In}{incolor}{11}{\boxspacing}
\begin{Verbatim}[commandchars=\\\{\}]
\PY{c+c1}{\PYZsh{} return which value is nul or not for each element in DataSet }
\PY{n}{df}\PY{o}{.}\PY{n}{isnull}\PY{p}{(}\PY{p}{)}
\end{Verbatim}
\end{tcolorbox}

            \begin{tcolorbox}[breakable, size=fbox, boxrule=.5pt, pad at break*=1mm, opacityfill=0]
\prompt{Out}{outcolor}{11}{\boxspacing}
\begin{Verbatim}[commandchars=\\\{\}]
        duration\_sec  start\_time  end\_time  start\_station\_id  \textbackslash{}
0              False       False     False             False
1              False       False     False             False
2              False       False     False             False
3              False       False     False             False
4              False       False     False             False
{\ldots}              {\ldots}         {\ldots}       {\ldots}               {\ldots}
183407         False       False     False             False
183408         False       False     False             False
183409         False       False     False             False
183410         False       False     False             False
183411         False       False     False             False

        start\_station\_name  start\_station\_latitude  start\_station\_longitude  \textbackslash{}
0                    False                   False                    False
1                    False                   False                    False
2                    False                   False                    False
3                    False                   False                    False
4                    False                   False                    False
{\ldots}                    {\ldots}                     {\ldots}                      {\ldots}
183407               False                   False                    False
183408               False                   False                    False
183409               False                   False                    False
183410               False                   False                    False
183411               False                   False                    False

        end\_station\_id  end\_station\_name  end\_station\_latitude  \textbackslash{}
0                False             False                 False
1                False             False                 False
2                False             False                 False
3                False             False                 False
4                False             False                 False
{\ldots}                {\ldots}               {\ldots}                   {\ldots}
183407           False             False                 False
183408           False             False                 False
183409           False             False                 False
183410           False             False                 False
183411           False             False                 False

        end\_station\_longitude  bike\_id  user\_type  member\_birth\_year  \textbackslash{}
0                       False    False      False              False
1                       False    False      False               True
2                       False    False      False              False
3                       False    False      False              False
4                       False    False      False              False
{\ldots}                       {\ldots}      {\ldots}        {\ldots}                {\ldots}
183407                  False    False      False              False
183408                  False    False      False              False
183409                  False    False      False              False
183410                  False    False      False              False
183411                  False    False      False              False

        member\_gender  bike\_share\_for\_all\_trip
0               False                    False
1                True                    False
2               False                    False
3               False                    False
4               False                    False
{\ldots}               {\ldots}                      {\ldots}
183407          False                    False
183408          False                    False
183409          False                    False
183410          False                    False
183411          False                    False

[183412 rows x 16 columns]
\end{Verbatim}
\end{tcolorbox}
        
    \begin{tcolorbox}[breakable, size=fbox, boxrule=1pt, pad at break*=1mm,colback=cellbackground, colframe=cellborder]
\prompt{In}{incolor}{12}{\boxspacing}
\begin{Verbatim}[commandchars=\\\{\}]
\PY{c+c1}{\PYZsh{} return which value is nul or not for each columns in DataSet }
\PY{n}{df}\PY{o}{.}\PY{n}{isnull}\PY{p}{(}\PY{p}{)}\PY{o}{.}\PY{n}{any}\PY{p}{(}\PY{p}{)}
\end{Verbatim}
\end{tcolorbox}

            \begin{tcolorbox}[breakable, size=fbox, boxrule=.5pt, pad at break*=1mm, opacityfill=0]
\prompt{Out}{outcolor}{12}{\boxspacing}
\begin{Verbatim}[commandchars=\\\{\}]
duration\_sec               False
start\_time                 False
end\_time                   False
start\_station\_id            True
start\_station\_name          True
start\_station\_latitude     False
start\_station\_longitude    False
end\_station\_id              True
end\_station\_name            True
end\_station\_latitude       False
end\_station\_longitude      False
bike\_id                    False
user\_type                  False
member\_birth\_year           True
member\_gender               True
bike\_share\_for\_all\_trip    False
dtype: bool
\end{Verbatim}
\end{tcolorbox}
        
    isnull().any() function show us boolean value for each column it is null
or not

if column null return True

else return False

    \begin{tcolorbox}[breakable, size=fbox, boxrule=1pt, pad at break*=1mm,colback=cellbackground, colframe=cellborder]
\prompt{In}{incolor}{13}{\boxspacing}
\begin{Verbatim}[commandchars=\\\{\}]
\PY{c+c1}{\PYZsh{}return number of columns has a null value}
\PY{n}{df}\PY{o}{.}\PY{n}{isnull}\PY{p}{(}\PY{p}{)}\PY{o}{.}\PY{n}{any}\PY{p}{(}\PY{p}{)}\PY{o}{.}\PY{n}{sum}\PY{p}{(}\PY{p}{)}
\end{Verbatim}
\end{tcolorbox}

            \begin{tcolorbox}[breakable, size=fbox, boxrule=.5pt, pad at break*=1mm, opacityfill=0]
\prompt{Out}{outcolor}{13}{\boxspacing}
\begin{Verbatim}[commandchars=\\\{\}]
6
\end{Verbatim}
\end{tcolorbox}
        
    \begin{tcolorbox}[breakable, size=fbox, boxrule=1pt, pad at break*=1mm,colback=cellbackground, colframe=cellborder]
\prompt{In}{incolor}{14}{\boxspacing}
\begin{Verbatim}[commandchars=\\\{\}]
\PY{c+c1}{\PYZsh{}return number of null value for each column}
\PY{n}{df}\PY{o}{.}\PY{n}{isnull}\PY{p}{(}\PY{p}{)}\PY{o}{.}\PY{n}{sum}\PY{p}{(}\PY{p}{)}
\end{Verbatim}
\end{tcolorbox}

            \begin{tcolorbox}[breakable, size=fbox, boxrule=.5pt, pad at break*=1mm, opacityfill=0]
\prompt{Out}{outcolor}{14}{\boxspacing}
\begin{Verbatim}[commandchars=\\\{\}]
duration\_sec                  0
start\_time                    0
end\_time                      0
start\_station\_id            197
start\_station\_name          197
start\_station\_latitude        0
start\_station\_longitude       0
end\_station\_id              197
end\_station\_name            197
end\_station\_latitude          0
end\_station\_longitude         0
bike\_id                       0
user\_type                     0
member\_birth\_year          8265
member\_gender              8265
bike\_share\_for\_all\_trip       0
dtype: int64
\end{Verbatim}
\end{tcolorbox}
        
    isnull().any() function show us boolean value for each column it is null
or not

if column null return 1

else return 0

    \begin{tcolorbox}[breakable, size=fbox, boxrule=1pt, pad at break*=1mm,colback=cellbackground, colframe=cellborder]
\prompt{In}{incolor}{15}{\boxspacing}
\begin{Verbatim}[commandchars=\\\{\}]
\PY{c+c1}{\PYZsh{}return a number of cell has a null value}
\PY{n}{df}\PY{o}{.}\PY{n}{isnull}\PY{p}{(}\PY{p}{)}\PY{o}{.}\PY{n}{sum}\PY{p}{(}\PY{p}{)}\PY{o}{.}\PY{n}{sum}\PY{p}{(}\PY{p}{)}
\end{Verbatim}
\end{tcolorbox}

            \begin{tcolorbox}[breakable, size=fbox, boxrule=.5pt, pad at break*=1mm, opacityfill=0]
\prompt{Out}{outcolor}{15}{\boxspacing}
\begin{Verbatim}[commandchars=\\\{\}]
17318
\end{Verbatim}
\end{tcolorbox}
        
    \begin{tcolorbox}[breakable, size=fbox, boxrule=1pt, pad at break*=1mm,colback=cellbackground, colframe=cellborder]
\prompt{In}{incolor}{16}{\boxspacing}
\begin{Verbatim}[commandchars=\\\{\}]
\PY{c+c1}{\PYZsh{}return statistical descriptive of dataset for each column}
\PY{n}{df}\PY{o}{.}\PY{n}{describe}\PY{p}{(}\PY{p}{)}
\end{Verbatim}
\end{tcolorbox}

            \begin{tcolorbox}[breakable, size=fbox, boxrule=.5pt, pad at break*=1mm, opacityfill=0]
\prompt{Out}{outcolor}{16}{\boxspacing}
\begin{Verbatim}[commandchars=\\\{\}]
        duration\_sec  start\_station\_id  start\_station\_latitude  \textbackslash{}
count  183412.000000     183215.000000           183412.000000
mean      726.078435        138.590427               37.771223
std      1794.389780        111.778864                0.099581
min        61.000000          3.000000               37.317298
25\%       325.000000         47.000000               37.770083
50\%       514.000000        104.000000               37.780760
75\%       796.000000        239.000000               37.797280
max     85444.000000        398.000000               37.880222

       start\_station\_longitude  end\_station\_id  end\_station\_latitude  \textbackslash{}
count            183412.000000   183215.000000         183412.000000
mean               -122.352664      136.249123             37.771427
std                   0.117097      111.515131              0.099490
min                -122.453704        3.000000             37.317298
25\%                -122.412408       44.000000             37.770407
50\%                -122.398285      100.000000             37.781010
75\%                -122.286533      235.000000             37.797320
max                -121.874119      398.000000             37.880222

       end\_station\_longitude        bike\_id  member\_birth\_year
count          183412.000000  183412.000000      175147.000000
mean             -122.352250    4472.906375        1984.806437
std                 0.116673    1664.383394          10.116689
min              -122.453704      11.000000        1878.000000
25\%              -122.411726    3777.000000        1980.000000
50\%              -122.398279    4958.000000        1987.000000
75\%              -122.288045    5502.000000        1992.000000
max              -121.874119    6645.000000        2001.000000
\end{Verbatim}
\end{tcolorbox}
        
    describe() function show us descriptive statistical value for each
column

in 8 value such as: count element in each column

\begin{verbatim}
                mean : getting average in each column
                
                std : standarded deviation
                    
                min : minimum value in each column
                    
                max : maximum value in each column
                    
                50% : median of value for each column
\end{verbatim}

    \hypertarget{data-cleaning}{%
\subsubsection{Data Cleaning}\label{data-cleaning}}

\begin{quote}
\textbf{Tip}: Make sure that you keep your reader informed on the steps
that you are taking in your investigation. Follow every code cell, or
every set of related code cells, with a markdown cell to describe to the
reader what was found in the preceding cell(s). Try to make it so that
the reader can then understand what they will be seeing in the following
cell(s).
\end{quote}

    1.duplicated data

2.missing value

3.incorrect datatype

    \begin{tcolorbox}[breakable, size=fbox, boxrule=1pt, pad at break*=1mm,colback=cellbackground, colframe=cellborder]
\prompt{In}{incolor}{17}{\boxspacing}
\begin{Verbatim}[commandchars=\\\{\}]
\PY{n}{df\PYZus{}clean} \PY{o}{=} \PY{n}{df}\PY{o}{.}\PY{n}{copy}\PY{p}{(}\PY{p}{)}
\end{Verbatim}
\end{tcolorbox}

    from assessing no NULL value

we will check for missing value and incorrect datatype

    \begin{tcolorbox}[breakable, size=fbox, boxrule=1pt, pad at break*=1mm,colback=cellbackground, colframe=cellborder]
\prompt{In}{incolor}{18}{\boxspacing}
\begin{Verbatim}[commandchars=\\\{\}]
\PY{c+c1}{\PYZsh{}\PYZsh{}df\PYZus{}clean.drop([\PYZsq{}PatientId\PYZsq{},\PYZsq{}AppointmentID\PYZsq{}],axis=1,inplace = True)}
\end{Verbatim}
\end{tcolorbox}

    \begin{tcolorbox}[breakable, size=fbox, boxrule=1pt, pad at break*=1mm,colback=cellbackground, colframe=cellborder]
\prompt{In}{incolor}{19}{\boxspacing}
\begin{Verbatim}[commandchars=\\\{\}]
\PY{n}{df\PYZus{}clean}\PY{o}{.}\PY{n}{drop}\PY{p}{(}\PY{n}{df\PYZus{}clean}\PY{p}{[} \PY{n}{df\PYZus{}clean}\PY{p}{[}\PY{l+s+s1}{\PYZsq{}}\PY{l+s+s1}{member\PYZus{}gender}\PY{l+s+s1}{\PYZsq{}}\PY{p}{]} \PY{o}{==} \PY{l+s+s1}{\PYZsq{}}\PY{l+s+s1}{Other}\PY{l+s+s1}{\PYZsq{}}\PY{p}{]}\PY{o}{.}\PY{n}{index}\PY{p}{,}\PY{n}{axis} \PY{o}{=} \PY{l+m+mi}{0}\PY{p}{,}\PY{n}{inplace} \PY{o}{=} \PY{k+kc}{True}\PY{p}{)}
\end{Verbatim}
\end{tcolorbox}

    \begin{tcolorbox}[breakable, size=fbox, boxrule=1pt, pad at break*=1mm,colback=cellbackground, colframe=cellborder]
\prompt{In}{incolor}{20}{\boxspacing}
\begin{Verbatim}[commandchars=\\\{\}]
\PY{n}{df\PYZus{}clean}\PY{o}{.}\PY{n}{head}\PY{p}{(}\PY{l+m+mi}{1}\PY{p}{)}
\end{Verbatim}
\end{tcolorbox}

            \begin{tcolorbox}[breakable, size=fbox, boxrule=.5pt, pad at break*=1mm, opacityfill=0]
\prompt{Out}{outcolor}{20}{\boxspacing}
\begin{Verbatim}[commandchars=\\\{\}]
   duration\_sec                start\_time                  end\_time  \textbackslash{}
0         52185  2019-02-28 17:32:10.1450  2019-03-01 08:01:55.9750

   start\_station\_id                                start\_station\_name  \textbackslash{}
0              21.0  Montgomery St BART Station (Market St at 2nd St)

   start\_station\_latitude  start\_station\_longitude  end\_station\_id  \textbackslash{}
0               37.789625              -122.400811            13.0

                 end\_station\_name  end\_station\_latitude  \textbackslash{}
0  Commercial St at Montgomery St             37.794231

   end\_station\_longitude  bike\_id user\_type  member\_birth\_year member\_gender  \textbackslash{}
0            -122.402923     4902  Customer             1984.0          Male

  bike\_share\_for\_all\_trip
0                      No
\end{Verbatim}
\end{tcolorbox}
        
    \begin{tcolorbox}[breakable, size=fbox, boxrule=1pt, pad at break*=1mm,colback=cellbackground, colframe=cellborder]
\prompt{In}{incolor}{21}{\boxspacing}
\begin{Verbatim}[commandchars=\\\{\}]
\PY{n}{df\PYZus{}clean}\PY{o}{.}\PY{n}{dtypes}
\end{Verbatim}
\end{tcolorbox}

            \begin{tcolorbox}[breakable, size=fbox, boxrule=.5pt, pad at break*=1mm, opacityfill=0]
\prompt{Out}{outcolor}{21}{\boxspacing}
\begin{Verbatim}[commandchars=\\\{\}]
duration\_sec                 int64
start\_time                  object
end\_time                    object
start\_station\_id           float64
start\_station\_name          object
start\_station\_latitude     float64
start\_station\_longitude    float64
end\_station\_id             float64
end\_station\_name            object
end\_station\_latitude       float64
end\_station\_longitude      float64
bike\_id                      int64
user\_type                   object
member\_birth\_year          float64
member\_gender               object
bike\_share\_for\_all\_trip     object
dtype: object
\end{Verbatim}
\end{tcolorbox}
        
    After discussing the structure of the data and any problems that need to
be

cleaned, perform those cleaning steps in the second part of this
section.

    \hypertarget{duplicicated-data}{%
\paragraph{duplicicated data}\label{duplicicated-data}}

    \begin{tcolorbox}[breakable, size=fbox, boxrule=1pt, pad at break*=1mm,colback=cellbackground, colframe=cellborder]
\prompt{In}{incolor}{22}{\boxspacing}
\begin{Verbatim}[commandchars=\\\{\}]
\PY{c+c1}{\PYZsh{}check for duplicated data}
\PY{n+nb}{sum}\PY{p}{(}\PY{n}{df\PYZus{}clean}\PY{o}{.}\PY{n}{duplicated}\PY{p}{(}\PY{p}{)}\PY{p}{)}
\end{Verbatim}
\end{tcolorbox}

            \begin{tcolorbox}[breakable, size=fbox, boxrule=.5pt, pad at break*=1mm, opacityfill=0]
\prompt{Out}{outcolor}{22}{\boxspacing}
\begin{Verbatim}[commandchars=\\\{\}]
0
\end{Verbatim}
\end{tcolorbox}
        
    \begin{tcolorbox}[breakable, size=fbox, boxrule=1pt, pad at break*=1mm,colback=cellbackground, colframe=cellborder]
\prompt{In}{incolor}{23}{\boxspacing}
\begin{Verbatim}[commandchars=\\\{\}]
\PY{c+c1}{\PYZsh{} if we have duplicated data we remove it}
\PY{n}{df\PYZus{}clean}\PY{o}{.}\PY{n}{drop\PYZus{}duplicates}\PY{p}{(}\PY{n}{inplace} \PY{o}{=} \PY{k+kc}{True}\PY{p}{)}
\end{Verbatim}
\end{tcolorbox}

    \begin{tcolorbox}[breakable, size=fbox, boxrule=1pt, pad at break*=1mm,colback=cellbackground, colframe=cellborder]
\prompt{In}{incolor}{24}{\boxspacing}
\begin{Verbatim}[commandchars=\\\{\}]
\PY{n+nb}{sum}\PY{p}{(}\PY{n}{df\PYZus{}clean}\PY{o}{.}\PY{n}{duplicated}\PY{p}{(}\PY{p}{)}\PY{p}{)}
\end{Verbatim}
\end{tcolorbox}

            \begin{tcolorbox}[breakable, size=fbox, boxrule=.5pt, pad at break*=1mm, opacityfill=0]
\prompt{Out}{outcolor}{24}{\boxspacing}
\begin{Verbatim}[commandchars=\\\{\}]
0
\end{Verbatim}
\end{tcolorbox}
        
    \hypertarget{missing-value}{%
\paragraph{Missing value}\label{missing-value}}

    \hypertarget{checking}{%
\subsubsection{checking}\label{checking}}

    \begin{tcolorbox}[breakable, size=fbox, boxrule=1pt, pad at break*=1mm,colback=cellbackground, colframe=cellborder]
\prompt{In}{incolor}{25}{\boxspacing}
\begin{Verbatim}[commandchars=\\\{\}]
\PY{n}{df\PYZus{}clean}\PY{o}{.}\PY{n}{head}\PY{p}{(}\PY{l+m+mi}{1}\PY{p}{)}
\end{Verbatim}
\end{tcolorbox}

            \begin{tcolorbox}[breakable, size=fbox, boxrule=.5pt, pad at break*=1mm, opacityfill=0]
\prompt{Out}{outcolor}{25}{\boxspacing}
\begin{Verbatim}[commandchars=\\\{\}]
   duration\_sec                start\_time                  end\_time  \textbackslash{}
0         52185  2019-02-28 17:32:10.1450  2019-03-01 08:01:55.9750

   start\_station\_id                                start\_station\_name  \textbackslash{}
0              21.0  Montgomery St BART Station (Market St at 2nd St)

   start\_station\_latitude  start\_station\_longitude  end\_station\_id  \textbackslash{}
0               37.789625              -122.400811            13.0

                 end\_station\_name  end\_station\_latitude  \textbackslash{}
0  Commercial St at Montgomery St             37.794231

   end\_station\_longitude  bike\_id user\_type  member\_birth\_year member\_gender  \textbackslash{}
0            -122.402923     4902  Customer             1984.0          Male

  bike\_share\_for\_all\_trip
0                      No
\end{Verbatim}
\end{tcolorbox}
        
    \begin{tcolorbox}[breakable, size=fbox, boxrule=1pt, pad at break*=1mm,colback=cellbackground, colframe=cellborder]
\prompt{In}{incolor}{26}{\boxspacing}
\begin{Verbatim}[commandchars=\\\{\}]
\PY{n}{df\PYZus{}clean}\PY{o}{.}\PY{n}{info}\PY{p}{(}\PY{p}{)}
\end{Verbatim}
\end{tcolorbox}

    \begin{Verbatim}[commandchars=\\\{\}]
<class 'pandas.core.frame.DataFrame'>
Int64Index: 179760 entries, 0 to 183411
Data columns (total 16 columns):
 \#   Column                   Non-Null Count   Dtype
---  ------                   --------------   -----
 0   duration\_sec             179760 non-null  int64
 1   start\_time               179760 non-null  object
 2   end\_time                 179760 non-null  object
 3   start\_station\_id         179568 non-null  float64
 4   start\_station\_name       179568 non-null  object
 5   start\_station\_latitude   179760 non-null  float64
 6   start\_station\_longitude  179760 non-null  float64
 7   end\_station\_id           179568 non-null  float64
 8   end\_station\_name         179568 non-null  object
 9   end\_station\_latitude     179760 non-null  float64
 10  end\_station\_longitude    179760 non-null  float64
 11  bike\_id                  179760 non-null  int64
 12  user\_type                179760 non-null  object
 13  member\_birth\_year        171495 non-null  float64
 14  member\_gender            171495 non-null  object
 15  bike\_share\_for\_all\_trip  179760 non-null  object
dtypes: float64(7), int64(2), object(7)
memory usage: 23.3+ MB
    \end{Verbatim}

    \begin{tcolorbox}[breakable, size=fbox, boxrule=1pt, pad at break*=1mm,colback=cellbackground, colframe=cellborder]
\prompt{In}{incolor}{27}{\boxspacing}
\begin{Verbatim}[commandchars=\\\{\}]
\PY{n}{df\PYZus{}clean}\PY{o}{.}\PY{n}{isnull}\PY{p}{(}\PY{p}{)}\PY{o}{.}\PY{n}{sum}\PY{p}{(}\PY{p}{)}
\end{Verbatim}
\end{tcolorbox}

            \begin{tcolorbox}[breakable, size=fbox, boxrule=.5pt, pad at break*=1mm, opacityfill=0]
\prompt{Out}{outcolor}{27}{\boxspacing}
\begin{Verbatim}[commandchars=\\\{\}]
duration\_sec                  0
start\_time                    0
end\_time                      0
start\_station\_id            192
start\_station\_name          192
start\_station\_latitude        0
start\_station\_longitude       0
end\_station\_id              192
end\_station\_name            192
end\_station\_latitude          0
end\_station\_longitude         0
bike\_id                       0
user\_type                     0
member\_birth\_year          8265
member\_gender              8265
bike\_share\_for\_all\_trip       0
dtype: int64
\end{Verbatim}
\end{tcolorbox}
        
    \begin{tcolorbox}[breakable, size=fbox, boxrule=1pt, pad at break*=1mm,colback=cellbackground, colframe=cellborder]
\prompt{In}{incolor}{28}{\boxspacing}
\begin{Verbatim}[commandchars=\\\{\}]
\PY{n}{df\PYZus{}clean}\PY{o}{.}\PY{n}{isnull}\PY{p}{(}\PY{p}{)}\PY{o}{.}\PY{n}{sum}\PY{p}{(}\PY{p}{)}\PY{o}{.}\PY{n}{sum}\PY{p}{(}\PY{p}{)}
\end{Verbatim}
\end{tcolorbox}

            \begin{tcolorbox}[breakable, size=fbox, boxrule=.5pt, pad at break*=1mm, opacityfill=0]
\prompt{Out}{outcolor}{28}{\boxspacing}
\begin{Verbatim}[commandchars=\\\{\}]
17298
\end{Verbatim}
\end{tcolorbox}
        
    \begin{tcolorbox}[breakable, size=fbox, boxrule=1pt, pad at break*=1mm,colback=cellbackground, colframe=cellborder]
\prompt{In}{incolor}{29}{\boxspacing}
\begin{Verbatim}[commandchars=\\\{\}]
\PY{n}{df\PYZus{}clean}\PY{o}{.}\PY{n}{dropna}\PY{p}{(}\PY{n}{inplace} \PY{o}{=} \PY{k+kc}{True}\PY{p}{)}
\end{Verbatim}
\end{tcolorbox}

    \begin{tcolorbox}[breakable, size=fbox, boxrule=1pt, pad at break*=1mm,colback=cellbackground, colframe=cellborder]
\prompt{In}{incolor}{30}{\boxspacing}
\begin{Verbatim}[commandchars=\\\{\}]
\PY{n}{df\PYZus{}clean}\PY{o}{.}\PY{n}{isnull}\PY{p}{(}\PY{p}{)}\PY{o}{.}\PY{n}{sum}\PY{p}{(}\PY{p}{)}
\end{Verbatim}
\end{tcolorbox}

            \begin{tcolorbox}[breakable, size=fbox, boxrule=.5pt, pad at break*=1mm, opacityfill=0]
\prompt{Out}{outcolor}{30}{\boxspacing}
\begin{Verbatim}[commandchars=\\\{\}]
duration\_sec               0
start\_time                 0
end\_time                   0
start\_station\_id           0
start\_station\_name         0
start\_station\_latitude     0
start\_station\_longitude    0
end\_station\_id             0
end\_station\_name           0
end\_station\_latitude       0
end\_station\_longitude      0
bike\_id                    0
user\_type                  0
member\_birth\_year          0
member\_gender              0
bike\_share\_for\_all\_trip    0
dtype: int64
\end{Verbatim}
\end{tcolorbox}
        
    \begin{tcolorbox}[breakable, size=fbox, boxrule=1pt, pad at break*=1mm,colback=cellbackground, colframe=cellborder]
\prompt{In}{incolor}{31}{\boxspacing}
\begin{Verbatim}[commandchars=\\\{\}]
\PY{n}{df\PYZus{}clean}\PY{o}{.}\PY{n}{isnull}\PY{p}{(}\PY{p}{)}\PY{o}{.}\PY{n}{sum}\PY{p}{(}\PY{p}{)}\PY{o}{.}\PY{n}{sum}\PY{p}{(}\PY{p}{)}
\end{Verbatim}
\end{tcolorbox}

            \begin{tcolorbox}[breakable, size=fbox, boxrule=.5pt, pad at break*=1mm, opacityfill=0]
\prompt{Out}{outcolor}{31}{\boxspacing}
\begin{Verbatim}[commandchars=\\\{\}]
0
\end{Verbatim}
\end{tcolorbox}
        
    \hypertarget{incorrect-datatype}{%
\subparagraph{incorrect datatype}\label{incorrect-datatype}}

    \begin{tcolorbox}[breakable, size=fbox, boxrule=1pt, pad at break*=1mm,colback=cellbackground, colframe=cellborder]
\prompt{In}{incolor}{32}{\boxspacing}
\begin{Verbatim}[commandchars=\\\{\}]
\PY{n}{df\PYZus{}clean}\PY{o}{.}\PY{n}{dtypes}
\end{Verbatim}
\end{tcolorbox}

            \begin{tcolorbox}[breakable, size=fbox, boxrule=.5pt, pad at break*=1mm, opacityfill=0]
\prompt{Out}{outcolor}{32}{\boxspacing}
\begin{Verbatim}[commandchars=\\\{\}]
duration\_sec                 int64
start\_time                  object
end\_time                    object
start\_station\_id           float64
start\_station\_name          object
start\_station\_latitude     float64
start\_station\_longitude    float64
end\_station\_id             float64
end\_station\_name            object
end\_station\_latitude       float64
end\_station\_longitude      float64
bike\_id                      int64
user\_type                   object
member\_birth\_year          float64
member\_gender               object
bike\_share\_for\_all\_trip     object
dtype: object
\end{Verbatim}
\end{tcolorbox}
        
    \begin{tcolorbox}[breakable, size=fbox, boxrule=1pt, pad at break*=1mm,colback=cellbackground, colframe=cellborder]
\prompt{In}{incolor}{33}{\boxspacing}
\begin{Verbatim}[commandchars=\\\{\}]
\PY{n}{df}\PY{p}{[}\PY{l+s+s1}{\PYZsq{}}\PY{l+s+s1}{start\PYZus{}time}\PY{l+s+s1}{\PYZsq{}}\PY{p}{]} \PY{o}{=} \PY{n}{pd}\PY{o}{.}\PY{n}{to\PYZus{}datetime}\PY{p}{(}\PY{n}{df}\PY{p}{[}\PY{l+s+s1}{\PYZsq{}}\PY{l+s+s1}{start\PYZus{}time}\PY{l+s+s1}{\PYZsq{}}\PY{p}{]}\PY{p}{)}
\end{Verbatim}
\end{tcolorbox}

    \begin{tcolorbox}[breakable, size=fbox, boxrule=1pt, pad at break*=1mm,colback=cellbackground, colframe=cellborder]
\prompt{In}{incolor}{34}{\boxspacing}
\begin{Verbatim}[commandchars=\\\{\}]
\PY{n}{df}\PY{p}{[}\PY{l+s+s1}{\PYZsq{}}\PY{l+s+s1}{end\PYZus{}time}\PY{l+s+s1}{\PYZsq{}}\PY{p}{]} \PY{o}{=} \PY{n}{pd}\PY{o}{.}\PY{n}{to\PYZus{}datetime}\PY{p}{(}\PY{n}{df}\PY{p}{[}\PY{l+s+s1}{\PYZsq{}}\PY{l+s+s1}{end\PYZus{}time}\PY{l+s+s1}{\PYZsq{}}\PY{p}{]}\PY{p}{)}
\end{Verbatim}
\end{tcolorbox}

    \begin{tcolorbox}[breakable, size=fbox, boxrule=1pt, pad at break*=1mm,colback=cellbackground, colframe=cellborder]
\prompt{In}{incolor}{35}{\boxspacing}
\begin{Verbatim}[commandchars=\\\{\}]
\PY{n}{df}\PY{p}{[}\PY{p}{[}\PY{l+s+s1}{\PYZsq{}}\PY{l+s+s1}{start\PYZus{}station\PYZus{}id}\PY{l+s+s1}{\PYZsq{}}\PY{p}{,}\PY{l+s+s1}{\PYZsq{}}\PY{l+s+s1}{end\PYZus{}station\PYZus{}id}\PY{l+s+s1}{\PYZsq{}}\PY{p}{,}\PY{l+s+s1}{\PYZsq{}}\PY{l+s+s1}{bike\PYZus{}id}\PY{l+s+s1}{\PYZsq{}}\PY{p}{,}\PY{l+s+s1}{\PYZsq{}}\PY{l+s+s1}{member\PYZus{}birth\PYZus{}year}\PY{l+s+s1}{\PYZsq{}}\PY{p}{]}\PY{p}{]}\PY{o}{=} \PY{n}{df}\PY{p}{[}\PY{p}{[}\PY{l+s+s1}{\PYZsq{}}\PY{l+s+s1}{start\PYZus{}station\PYZus{}id}\PY{l+s+s1}{\PYZsq{}}\PY{p}{,}\PY{l+s+s1}{\PYZsq{}}\PY{l+s+s1}{end\PYZus{}station\PYZus{}id}\PY{l+s+s1}{\PYZsq{}}\PY{p}{,}\PY{l+s+s1}{\PYZsq{}}\PY{l+s+s1}{bike\PYZus{}id}\PY{l+s+s1}{\PYZsq{}}\PY{p}{,}\PY{l+s+s1}{\PYZsq{}}\PY{l+s+s1}{member\PYZus{}birth\PYZus{}year}\PY{l+s+s1}{\PYZsq{}}\PY{p}{]}\PY{p}{]}\PY{o}{.}\PY{n}{astype}\PY{p}{(}\PY{n+nb}{str}\PY{p}{)}
\end{Verbatim}
\end{tcolorbox}

    \hypertarget{checking}{%
\subsubsection{checking}\label{checking}}

    \begin{tcolorbox}[breakable, size=fbox, boxrule=1pt, pad at break*=1mm,colback=cellbackground, colframe=cellborder]
\prompt{In}{incolor}{36}{\boxspacing}
\begin{Verbatim}[commandchars=\\\{\}]
\PY{n}{df\PYZus{}clean}\PY{o}{.}\PY{n}{dtypes}
\end{Verbatim}
\end{tcolorbox}

            \begin{tcolorbox}[breakable, size=fbox, boxrule=.5pt, pad at break*=1mm, opacityfill=0]
\prompt{Out}{outcolor}{36}{\boxspacing}
\begin{Verbatim}[commandchars=\\\{\}]
duration\_sec                 int64
start\_time                  object
end\_time                    object
start\_station\_id           float64
start\_station\_name          object
start\_station\_latitude     float64
start\_station\_longitude    float64
end\_station\_id             float64
end\_station\_name            object
end\_station\_latitude       float64
end\_station\_longitude      float64
bike\_id                      int64
user\_type                   object
member\_birth\_year          float64
member\_gender               object
bike\_share\_for\_all\_trip     object
dtype: object
\end{Verbatim}
\end{tcolorbox}
        
    \begin{tcolorbox}[breakable, size=fbox, boxrule=1pt, pad at break*=1mm,colback=cellbackground, colframe=cellborder]
\prompt{In}{incolor}{37}{\boxspacing}
\begin{Verbatim}[commandchars=\\\{\}]
\PY{n}{df\PYZus{}clean}\PY{o}{.}\PY{n}{head}\PY{p}{(}\PY{l+m+mi}{1}\PY{p}{)}
\end{Verbatim}
\end{tcolorbox}

            \begin{tcolorbox}[breakable, size=fbox, boxrule=.5pt, pad at break*=1mm, opacityfill=0]
\prompt{Out}{outcolor}{37}{\boxspacing}
\begin{Verbatim}[commandchars=\\\{\}]
   duration\_sec                start\_time                  end\_time  \textbackslash{}
0         52185  2019-02-28 17:32:10.1450  2019-03-01 08:01:55.9750

   start\_station\_id                                start\_station\_name  \textbackslash{}
0              21.0  Montgomery St BART Station (Market St at 2nd St)

   start\_station\_latitude  start\_station\_longitude  end\_station\_id  \textbackslash{}
0               37.789625              -122.400811            13.0

                 end\_station\_name  end\_station\_latitude  \textbackslash{}
0  Commercial St at Montgomery St             37.794231

   end\_station\_longitude  bike\_id user\_type  member\_birth\_year member\_gender  \textbackslash{}
0            -122.402923     4902  Customer             1984.0          Male

  bike\_share\_for\_all\_trip
0                      No
\end{Verbatim}
\end{tcolorbox}
        
    convert sec to minute

    \begin{tcolorbox}[breakable, size=fbox, boxrule=1pt, pad at break*=1mm,colback=cellbackground, colframe=cellborder]
\prompt{In}{incolor}{38}{\boxspacing}
\begin{Verbatim}[commandchars=\\\{\}]
\PY{n}{duration\PYZus{}minu} \PY{o}{=} \PY{n}{df\PYZus{}clean}\PY{o}{.}\PY{n}{duration\PYZus{}sec}\PY{o}{/}\PY{l+m+mi}{60}
\end{Verbatim}
\end{tcolorbox}

    \begin{tcolorbox}[breakable, size=fbox, boxrule=1pt, pad at break*=1mm,colback=cellbackground, colframe=cellborder]
\prompt{In}{incolor}{39}{\boxspacing}
\begin{Verbatim}[commandchars=\\\{\}]
\PY{n}{duration\PYZus{}minu}
\end{Verbatim}
\end{tcolorbox}

            \begin{tcolorbox}[breakable, size=fbox, boxrule=.5pt, pad at break*=1mm, opacityfill=0]
\prompt{Out}{outcolor}{39}{\boxspacing}
\begin{Verbatim}[commandchars=\\\{\}]
0          869.750000
2         1030.900000
4           26.416667
5           29.883333
6           19.116667
             {\ldots}
183407       8.000000
183408       5.216667
183409       2.350000
183410       2.316667
183411       4.516667
Name: duration\_sec, Length: 171305, dtype: float64
\end{Verbatim}
\end{tcolorbox}
        
    \begin{tcolorbox}[breakable, size=fbox, boxrule=1pt, pad at break*=1mm,colback=cellbackground, colframe=cellborder]
\prompt{In}{incolor}{40}{\boxspacing}
\begin{Verbatim}[commandchars=\\\{\}]
\PY{n}{df\PYZus{}clean}\PY{p}{[}\PY{l+s+s1}{\PYZsq{}}\PY{l+s+s1}{duration\PYZus{}minu}\PY{l+s+s1}{\PYZsq{}}\PY{p}{]} \PY{o}{=} \PY{n}{duration\PYZus{}minu}
\end{Verbatim}
\end{tcolorbox}

    checking

    \begin{tcolorbox}[breakable, size=fbox, boxrule=1pt, pad at break*=1mm,colback=cellbackground, colframe=cellborder]
\prompt{In}{incolor}{41}{\boxspacing}
\begin{Verbatim}[commandchars=\\\{\}]
\PY{n}{df\PYZus{}clean}\PY{o}{.}\PY{n}{head}\PY{p}{(}\PY{l+m+mi}{1}\PY{p}{)}
\end{Verbatim}
\end{tcolorbox}

            \begin{tcolorbox}[breakable, size=fbox, boxrule=.5pt, pad at break*=1mm, opacityfill=0]
\prompt{Out}{outcolor}{41}{\boxspacing}
\begin{Verbatim}[commandchars=\\\{\}]
   duration\_sec                start\_time                  end\_time  \textbackslash{}
0         52185  2019-02-28 17:32:10.1450  2019-03-01 08:01:55.9750

   start\_station\_id                                start\_station\_name  \textbackslash{}
0              21.0  Montgomery St BART Station (Market St at 2nd St)

   start\_station\_latitude  start\_station\_longitude  end\_station\_id  \textbackslash{}
0               37.789625              -122.400811            13.0

                 end\_station\_name  end\_station\_latitude  \textbackslash{}
0  Commercial St at Montgomery St             37.794231

   end\_station\_longitude  bike\_id user\_type  member\_birth\_year member\_gender  \textbackslash{}
0            -122.402923     4902  Customer             1984.0          Male

  bike\_share\_for\_all\_trip  duration\_minu
0                      No         869.75
\end{Verbatim}
\end{tcolorbox}
        
    convert minute to hours

    \begin{tcolorbox}[breakable, size=fbox, boxrule=1pt, pad at break*=1mm,colback=cellbackground, colframe=cellborder]
\prompt{In}{incolor}{42}{\boxspacing}
\begin{Verbatim}[commandchars=\\\{\}]
\PY{n}{duration\PYZus{}hr} \PY{o}{=} \PY{p}{(}\PY{n}{df\PYZus{}clean}\PY{o}{.}\PY{n}{duration\PYZus{}sec}\PY{o}{/}\PY{l+m+mi}{60}\PY{p}{)}\PY{o}{/}\PY{l+m+mi}{60}
\end{Verbatim}
\end{tcolorbox}

    \begin{tcolorbox}[breakable, size=fbox, boxrule=1pt, pad at break*=1mm,colback=cellbackground, colframe=cellborder]
\prompt{In}{incolor}{43}{\boxspacing}
\begin{Verbatim}[commandchars=\\\{\}]
\PY{n}{duration\PYZus{}hr}
\end{Verbatim}
\end{tcolorbox}

            \begin{tcolorbox}[breakable, size=fbox, boxrule=.5pt, pad at break*=1mm, opacityfill=0]
\prompt{Out}{outcolor}{43}{\boxspacing}
\begin{Verbatim}[commandchars=\\\{\}]
0         14.495833
2         17.181667
4          0.440278
5          0.498056
6          0.318611
            {\ldots}
183407     0.133333
183408     0.086944
183409     0.039167
183410     0.038611
183411     0.075278
Name: duration\_sec, Length: 171305, dtype: float64
\end{Verbatim}
\end{tcolorbox}
        
    \begin{tcolorbox}[breakable, size=fbox, boxrule=1pt, pad at break*=1mm,colback=cellbackground, colframe=cellborder]
\prompt{In}{incolor}{44}{\boxspacing}
\begin{Verbatim}[commandchars=\\\{\}]
\PY{n}{df\PYZus{}clean}\PY{p}{[}\PY{l+s+s1}{\PYZsq{}}\PY{l+s+s1}{duration\PYZus{}hr}\PY{l+s+s1}{\PYZsq{}}\PY{p}{]} \PY{o}{=} \PY{n}{duration\PYZus{}hr}
\end{Verbatim}
\end{tcolorbox}

    \begin{tcolorbox}[breakable, size=fbox, boxrule=1pt, pad at break*=1mm,colback=cellbackground, colframe=cellborder]
\prompt{In}{incolor}{45}{\boxspacing}
\begin{Verbatim}[commandchars=\\\{\}]
\PY{n}{df\PYZus{}clean}\PY{o}{.}\PY{n}{head}\PY{p}{(}\PY{l+m+mi}{1}\PY{p}{)}
\end{Verbatim}
\end{tcolorbox}

            \begin{tcolorbox}[breakable, size=fbox, boxrule=.5pt, pad at break*=1mm, opacityfill=0]
\prompt{Out}{outcolor}{45}{\boxspacing}
\begin{Verbatim}[commandchars=\\\{\}]
   duration\_sec                start\_time                  end\_time  \textbackslash{}
0         52185  2019-02-28 17:32:10.1450  2019-03-01 08:01:55.9750

   start\_station\_id                                start\_station\_name  \textbackslash{}
0              21.0  Montgomery St BART Station (Market St at 2nd St)

   start\_station\_latitude  start\_station\_longitude  end\_station\_id  \textbackslash{}
0               37.789625              -122.400811            13.0

                 end\_station\_name  end\_station\_latitude  \textbackslash{}
0  Commercial St at Montgomery St             37.794231

   end\_station\_longitude  bike\_id user\_type  member\_birth\_year member\_gender  \textbackslash{}
0            -122.402923     4902  Customer             1984.0          Male

  bike\_share\_for\_all\_trip  duration\_minu  duration\_hr
0                      No         869.75    14.495833
\end{Verbatim}
\end{tcolorbox}
        
    \hypertarget{convert-hours-to-days}{%
\subparagraph{convert hours to days}\label{convert-hours-to-days}}

    \begin{tcolorbox}[breakable, size=fbox, boxrule=1pt, pad at break*=1mm,colback=cellbackground, colframe=cellborder]
\prompt{In}{incolor}{46}{\boxspacing}
\begin{Verbatim}[commandchars=\\\{\}]
\PY{n}{duration\PYZus{}days} \PY{o}{=} \PY{p}{(}\PY{p}{(}\PY{n}{df\PYZus{}clean}\PY{o}{.}\PY{n}{duration\PYZus{}sec}\PY{o}{/}\PY{l+m+mi}{60}\PY{p}{)}\PY{o}{/}\PY{l+m+mi}{60}\PY{p}{)}\PY{o}{/}\PY{l+m+mi}{24}
\end{Verbatim}
\end{tcolorbox}

    \begin{tcolorbox}[breakable, size=fbox, boxrule=1pt, pad at break*=1mm,colback=cellbackground, colframe=cellborder]
\prompt{In}{incolor}{47}{\boxspacing}
\begin{Verbatim}[commandchars=\\\{\}]
\PY{n}{duration\PYZus{}days}
\end{Verbatim}
\end{tcolorbox}

            \begin{tcolorbox}[breakable, size=fbox, boxrule=.5pt, pad at break*=1mm, opacityfill=0]
\prompt{Out}{outcolor}{47}{\boxspacing}
\begin{Verbatim}[commandchars=\\\{\}]
0         0.603993
2         0.715903
4         0.018345
5         0.020752
6         0.013275
            {\ldots}
183407    0.005556
183408    0.003623
183409    0.001632
183410    0.001609
183411    0.003137
Name: duration\_sec, Length: 171305, dtype: float64
\end{Verbatim}
\end{tcolorbox}
        
    \begin{tcolorbox}[breakable, size=fbox, boxrule=1pt, pad at break*=1mm,colback=cellbackground, colframe=cellborder]
\prompt{In}{incolor}{48}{\boxspacing}
\begin{Verbatim}[commandchars=\\\{\}]
\PY{n}{df\PYZus{}clean}\PY{p}{[}\PY{l+s+s1}{\PYZsq{}}\PY{l+s+s1}{duration\PYZus{}days}\PY{l+s+s1}{\PYZsq{}}\PY{p}{]} \PY{o}{=} \PY{n}{duration\PYZus{}days}
\end{Verbatim}
\end{tcolorbox}

    \begin{tcolorbox}[breakable, size=fbox, boxrule=1pt, pad at break*=1mm,colback=cellbackground, colframe=cellborder]
\prompt{In}{incolor}{49}{\boxspacing}
\begin{Verbatim}[commandchars=\\\{\}]
\PY{n}{df\PYZus{}clean}\PY{o}{.}\PY{n}{head}\PY{p}{(}\PY{l+m+mi}{1}\PY{p}{)}
\end{Verbatim}
\end{tcolorbox}

            \begin{tcolorbox}[breakable, size=fbox, boxrule=.5pt, pad at break*=1mm, opacityfill=0]
\prompt{Out}{outcolor}{49}{\boxspacing}
\begin{Verbatim}[commandchars=\\\{\}]
   duration\_sec                start\_time                  end\_time  \textbackslash{}
0         52185  2019-02-28 17:32:10.1450  2019-03-01 08:01:55.9750

   start\_station\_id                                start\_station\_name  \textbackslash{}
0              21.0  Montgomery St BART Station (Market St at 2nd St)

   start\_station\_latitude  start\_station\_longitude  end\_station\_id  \textbackslash{}
0               37.789625              -122.400811            13.0

                 end\_station\_name  end\_station\_latitude  \textbackslash{}
0  Commercial St at Montgomery St             37.794231

   end\_station\_longitude  bike\_id user\_type  member\_birth\_year member\_gender  \textbackslash{}
0            -122.402923     4902  Customer             1984.0          Male

  bike\_share\_for\_all\_trip  duration\_minu  duration\_hr  duration\_days
0                      No         869.75    14.495833       0.603993
\end{Verbatim}
\end{tcolorbox}
        
    \hypertarget{convert-days-to-weeks}{%
\subparagraph{convert days to weeks}\label{convert-days-to-weeks}}

    \begin{tcolorbox}[breakable, size=fbox, boxrule=1pt, pad at break*=1mm,colback=cellbackground, colframe=cellborder]
\prompt{In}{incolor}{50}{\boxspacing}
\begin{Verbatim}[commandchars=\\\{\}]
\PY{n}{duration\PYZus{}weeks} \PY{o}{=} \PY{p}{(}\PY{p}{(}\PY{p}{(}\PY{n}{df\PYZus{}clean}\PY{o}{.}\PY{n}{duration\PYZus{}sec}\PY{o}{/}\PY{l+m+mi}{60}\PY{p}{)}\PY{o}{/}\PY{l+m+mi}{60}\PY{p}{)}\PY{o}{/}\PY{l+m+mi}{24}\PY{p}{)}\PY{o}{/}\PY{l+m+mi}{7}
\end{Verbatim}
\end{tcolorbox}

    \begin{tcolorbox}[breakable, size=fbox, boxrule=1pt, pad at break*=1mm,colback=cellbackground, colframe=cellborder]
\prompt{In}{incolor}{51}{\boxspacing}
\begin{Verbatim}[commandchars=\\\{\}]
\PY{n}{df\PYZus{}clean}\PY{p}{[}\PY{l+s+s1}{\PYZsq{}}\PY{l+s+s1}{duration\PYZus{}weeks}\PY{l+s+s1}{\PYZsq{}}\PY{p}{]} \PY{o}{=} \PY{n}{duration\PYZus{}weeks}
\end{Verbatim}
\end{tcolorbox}

    \begin{tcolorbox}[breakable, size=fbox, boxrule=1pt, pad at break*=1mm,colback=cellbackground, colframe=cellborder]
\prompt{In}{incolor}{52}{\boxspacing}
\begin{Verbatim}[commandchars=\\\{\}]
\PY{n}{df\PYZus{}clean}\PY{o}{.}\PY{n}{head}\PY{p}{(}\PY{l+m+mi}{1}\PY{p}{)}
\end{Verbatim}
\end{tcolorbox}

            \begin{tcolorbox}[breakable, size=fbox, boxrule=.5pt, pad at break*=1mm, opacityfill=0]
\prompt{Out}{outcolor}{52}{\boxspacing}
\begin{Verbatim}[commandchars=\\\{\}]
   duration\_sec                start\_time                  end\_time  \textbackslash{}
0         52185  2019-02-28 17:32:10.1450  2019-03-01 08:01:55.9750

   start\_station\_id                                start\_station\_name  \textbackslash{}
0              21.0  Montgomery St BART Station (Market St at 2nd St)

   start\_station\_latitude  start\_station\_longitude  end\_station\_id  \textbackslash{}
0               37.789625              -122.400811            13.0

                 end\_station\_name  end\_station\_latitude  \textbackslash{}
0  Commercial St at Montgomery St             37.794231

   end\_station\_longitude  bike\_id user\_type  member\_birth\_year member\_gender  \textbackslash{}
0            -122.402923     4902  Customer             1984.0          Male

  bike\_share\_for\_all\_trip  duration\_minu  duration\_hr  duration\_days  \textbackslash{}
0                      No         869.75    14.495833       0.603993

   duration\_weeks
0        0.086285
\end{Verbatim}
\end{tcolorbox}
        
    \hypertarget{convert-weeks-to-month}{%
\subparagraph{convert weeks to month}\label{convert-weeks-to-month}}

    \begin{tcolorbox}[breakable, size=fbox, boxrule=1pt, pad at break*=1mm,colback=cellbackground, colframe=cellborder]
\prompt{In}{incolor}{53}{\boxspacing}
\begin{Verbatim}[commandchars=\\\{\}]
\PY{n}{duration\PYZus{}months} \PY{o}{=} \PY{p}{(}\PY{p}{(}\PY{p}{(}\PY{p}{(}\PY{n}{df\PYZus{}clean}\PY{o}{.}\PY{n}{duration\PYZus{}sec}\PY{o}{/}\PY{l+m+mi}{60}\PY{p}{)}\PY{o}{/}\PY{l+m+mi}{60}\PY{p}{)}\PY{o}{/}\PY{l+m+mi}{24}\PY{p}{)}\PY{o}{/}\PY{l+m+mi}{7}\PY{p}{)}\PY{o}{/}\PY{l+m+mi}{4}
\end{Verbatim}
\end{tcolorbox}

    \begin{tcolorbox}[breakable, size=fbox, boxrule=1pt, pad at break*=1mm,colback=cellbackground, colframe=cellborder]
\prompt{In}{incolor}{54}{\boxspacing}
\begin{Verbatim}[commandchars=\\\{\}]
\PY{n}{df\PYZus{}clean}\PY{p}{[}\PY{l+s+s1}{\PYZsq{}}\PY{l+s+s1}{duration\PYZus{}months}\PY{l+s+s1}{\PYZsq{}}\PY{p}{]} \PY{o}{=} \PY{n}{duration\PYZus{}months}
\end{Verbatim}
\end{tcolorbox}

    \begin{tcolorbox}[breakable, size=fbox, boxrule=1pt, pad at break*=1mm,colback=cellbackground, colframe=cellborder]
\prompt{In}{incolor}{55}{\boxspacing}
\begin{Verbatim}[commandchars=\\\{\}]
\PY{n}{df\PYZus{}clean}\PY{o}{.}\PY{n}{head}\PY{p}{(}\PY{l+m+mi}{1}\PY{p}{)}
\end{Verbatim}
\end{tcolorbox}

            \begin{tcolorbox}[breakable, size=fbox, boxrule=.5pt, pad at break*=1mm, opacityfill=0]
\prompt{Out}{outcolor}{55}{\boxspacing}
\begin{Verbatim}[commandchars=\\\{\}]
   duration\_sec                start\_time                  end\_time  \textbackslash{}
0         52185  2019-02-28 17:32:10.1450  2019-03-01 08:01:55.9750

   start\_station\_id                                start\_station\_name  \textbackslash{}
0              21.0  Montgomery St BART Station (Market St at 2nd St)

   start\_station\_latitude  start\_station\_longitude  end\_station\_id  \textbackslash{}
0               37.789625              -122.400811            13.0

                 end\_station\_name  end\_station\_latitude  {\ldots}  bike\_id  \textbackslash{}
0  Commercial St at Montgomery St             37.794231  {\ldots}     4902

   user\_type member\_birth\_year  member\_gender bike\_share\_for\_all\_trip  \textbackslash{}
0   Customer            1984.0           Male                      No

  duration\_minu  duration\_hr  duration\_days  duration\_weeks  duration\_months
0        869.75    14.495833       0.603993        0.086285         0.021571

[1 rows x 21 columns]
\end{Verbatim}
\end{tcolorbox}
        
    \begin{tcolorbox}[breakable, size=fbox, boxrule=1pt, pad at break*=1mm,colback=cellbackground, colframe=cellborder]
\prompt{In}{incolor}{56}{\boxspacing}
\begin{Verbatim}[commandchars=\\\{\}]
\PY{n}{df\PYZus{}clean}\PY{o}{.}\PY{n}{to\PYZus{}csv}\PY{p}{(}\PY{l+s+s2}{\PYZdq{}}\PY{l+s+s2}{new data/Go\PYZus{}Bike.csv}\PY{l+s+s2}{\PYZdq{}}\PY{p}{)}\PY{p}{;}
\end{Verbatim}
\end{tcolorbox}

    \hypertarget{done}{%
\paragraph{DONE}\label{done}}

\hypertarget{non_dublicated-data.non_missing-value.non-incorrect-datatype}{%
\paragraph{non\_dublicated data\ldots.non\_missing value\ldots.non
incorrect
datatype}\label{non_dublicated-data.non_missing-value.non-incorrect-datatype}}

    \#\# Exploratory Data Analysis

\begin{quote}
\textbf{Tip}: Now that you've trimmed and cleaned your data, you're
ready to move on to exploration. \textbf{Compute statistics} and
\textbf{create visualizations} with the goal of addressing the research
questions that you posed in the Introduction section. You should compute
the relevant statistics throughout the analysis when an inference is
made about the data. Note that at least two or more kinds of plots
should be created as part of the exploration, and you must compare and
show trends in the varied visualizations.
\end{quote}

\begin{quote}
\textbf{Tip}: - Investigate the stated question(s) from multiple angles.
It is recommended that you be systematic with your approach. Look at one
variable at a time, and then follow it up by looking at relationships
between variables. You should explore at least three variables in
relation to the primary question. This can be an exploratory
relationship between three variables of interest, or looking at how two
independent variables relate to a single dependent variable of interest.
Lastly, you should perform both single-variable (1d) and
multiple-variable (2d) explorations.
\end{quote}

\hypertarget{research-question}{%
\subsubsection{Research Question:}\label{research-question}}

\begin{enumerate}
\def\labelenumi{\arabic{enumi})}
\item
  When are most trips taken in terms of time of day, day of the week, or
  month of the year?
\item
  How long does the average trip take?
\item
  Does the above depend on if a user is a subscriber or customer?
\end{enumerate}

    \begin{tcolorbox}[breakable, size=fbox, boxrule=1pt, pad at break*=1mm,colback=cellbackground, colframe=cellborder]
\prompt{In}{incolor}{57}{\boxspacing}
\begin{Verbatim}[commandchars=\\\{\}]
\PY{n}{df\PYZus{}go\PYZus{}bike} \PY{o}{=} \PY{n}{pd}\PY{o}{.}\PY{n}{read\PYZus{}csv}\PY{p}{(}\PY{l+s+s2}{\PYZdq{}}\PY{l+s+s2}{new data/Go\PYZus{}Bike.csv}\PY{l+s+s2}{\PYZdq{}}\PY{p}{)}
\end{Verbatim}
\end{tcolorbox}

    \begin{tcolorbox}[breakable, size=fbox, boxrule=1pt, pad at break*=1mm,colback=cellbackground, colframe=cellborder]
\prompt{In}{incolor}{58}{\boxspacing}
\begin{Verbatim}[commandchars=\\\{\}]
\PY{n}{df\PYZus{}go\PYZus{}bike}\PY{o}{.}\PY{n}{info}\PY{p}{(}\PY{p}{)}
\end{Verbatim}
\end{tcolorbox}

    \begin{Verbatim}[commandchars=\\\{\}]
<class 'pandas.core.frame.DataFrame'>
RangeIndex: 171305 entries, 0 to 171304
Data columns (total 22 columns):
 \#   Column                   Non-Null Count   Dtype
---  ------                   --------------   -----
 0   Unnamed: 0               171305 non-null  int64
 1   duration\_sec             171305 non-null  int64
 2   start\_time               171305 non-null  object
 3   end\_time                 171305 non-null  object
 4   start\_station\_id         171305 non-null  float64
 5   start\_station\_name       171305 non-null  object
 6   start\_station\_latitude   171305 non-null  float64
 7   start\_station\_longitude  171305 non-null  float64
 8   end\_station\_id           171305 non-null  float64
 9   end\_station\_name         171305 non-null  object
 10  end\_station\_latitude     171305 non-null  float64
 11  end\_station\_longitude    171305 non-null  float64
 12  bike\_id                  171305 non-null  int64
 13  user\_type                171305 non-null  object
 14  member\_birth\_year        171305 non-null  float64
 15  member\_gender            171305 non-null  object
 16  bike\_share\_for\_all\_trip  171305 non-null  object
 17  duration\_minu            171305 non-null  float64
 18  duration\_hr              171305 non-null  float64
 19  duration\_days            171305 non-null  float64
 20  duration\_weeks           171305 non-null  float64
 21  duration\_months          171305 non-null  float64
dtypes: float64(12), int64(3), object(7)
memory usage: 28.8+ MB
    \end{Verbatim}

    \begin{tcolorbox}[breakable, size=fbox, boxrule=1pt, pad at break*=1mm,colback=cellbackground, colframe=cellborder]
\prompt{In}{incolor}{59}{\boxspacing}
\begin{Verbatim}[commandchars=\\\{\}]
\PY{n}{df\PYZus{}go\PYZus{}bike}\PY{o}{.}\PY{n}{isnull}\PY{p}{(}\PY{p}{)}\PY{o}{.}\PY{n}{sum}\PY{p}{(}\PY{p}{)}\PY{o}{.}\PY{n}{sum}\PY{p}{(}\PY{p}{)}
\end{Verbatim}
\end{tcolorbox}

            \begin{tcolorbox}[breakable, size=fbox, boxrule=.5pt, pad at break*=1mm, opacityfill=0]
\prompt{Out}{outcolor}{59}{\boxspacing}
\begin{Verbatim}[commandchars=\\\{\}]
0
\end{Verbatim}
\end{tcolorbox}
        
    \hypertarget{univariate-exploration}{%
\subsection{Univariate Exploration}\label{univariate-exploration}}

I'll start by looking at the distribution of the main variable of
interest:

    \hypertarget{discuss-the-distributions-of-your-variables-of-interest.-were-there-any-unusual-points-did-you-need-to-perform-any-transformations}{%
\subsubsection{Discuss the distribution(s) of your variable(s) of
interest. Were there any unusual points? Did you need to perform any
transformations?}\label{discuss-the-distributions-of-your-variables-of-interest.-were-there-any-unusual-points-did-you-need-to-perform-any-transformations}}

using hist() function to draw Histogram for each column and use
figsize(,) parameter to show it obviously.

    \begin{tcolorbox}[breakable, size=fbox, boxrule=1pt, pad at break*=1mm,colback=cellbackground, colframe=cellborder]
\prompt{In}{incolor}{60}{\boxspacing}
\begin{Verbatim}[commandchars=\\\{\}]
\PY{n}{df\PYZus{}go\PYZus{}bike}\PY{o}{.}\PY{n}{info}\PY{p}{(}\PY{p}{)}
\end{Verbatim}
\end{tcolorbox}

    \begin{Verbatim}[commandchars=\\\{\}]
<class 'pandas.core.frame.DataFrame'>
RangeIndex: 171305 entries, 0 to 171304
Data columns (total 22 columns):
 \#   Column                   Non-Null Count   Dtype
---  ------                   --------------   -----
 0   Unnamed: 0               171305 non-null  int64
 1   duration\_sec             171305 non-null  int64
 2   start\_time               171305 non-null  object
 3   end\_time                 171305 non-null  object
 4   start\_station\_id         171305 non-null  float64
 5   start\_station\_name       171305 non-null  object
 6   start\_station\_latitude   171305 non-null  float64
 7   start\_station\_longitude  171305 non-null  float64
 8   end\_station\_id           171305 non-null  float64
 9   end\_station\_name         171305 non-null  object
 10  end\_station\_latitude     171305 non-null  float64
 11  end\_station\_longitude    171305 non-null  float64
 12  bike\_id                  171305 non-null  int64
 13  user\_type                171305 non-null  object
 14  member\_birth\_year        171305 non-null  float64
 15  member\_gender            171305 non-null  object
 16  bike\_share\_for\_all\_trip  171305 non-null  object
 17  duration\_minu            171305 non-null  float64
 18  duration\_hr              171305 non-null  float64
 19  duration\_days            171305 non-null  float64
 20  duration\_weeks           171305 non-null  float64
 21  duration\_months          171305 non-null  float64
dtypes: float64(12), int64(3), object(7)
memory usage: 28.8+ MB
    \end{Verbatim}

    \begin{tcolorbox}[breakable, size=fbox, boxrule=1pt, pad at break*=1mm,colback=cellbackground, colframe=cellborder]
\prompt{In}{incolor}{61}{\boxspacing}
\begin{Verbatim}[commandchars=\\\{\}]
\PY{k}{def} \PY{n+nf}{bar\PYZus{}plot} \PY{p}{(}\PY{n}{col\PYZus{}name}\PY{p}{)}\PY{p}{:}   
    \PY{c+c1}{\PYZsh{} Return the Series having unique values}
    \PY{n}{x} \PY{o}{=} \PY{n}{df\PYZus{}go\PYZus{}bike}\PY{p}{[}\PY{n}{col\PYZus{}name}\PY{p}{]}\PY{o}{.}\PY{n}{unique}\PY{p}{(}\PY{p}{)}

    \PY{c+c1}{\PYZsh{} Return the Series having frequency count of each unique value}
    \PY{n}{y} \PY{o}{=} \PY{n}{df\PYZus{}go\PYZus{}bike}\PY{p}{[}\PY{n}{col\PYZus{}name}\PY{p}{]}\PY{o}{.}\PY{n}{value\PYZus{}counts}\PY{p}{(}\PY{n}{sort}\PY{o}{=}\PY{k+kc}{False}\PY{p}{)}
    
    \PY{n}{plt}\PY{o}{.}\PY{n}{subplots}\PY{p}{(}\PY{n}{figsize}\PY{o}{=}\PY{p}{(}\PY{l+m+mi}{18}\PY{p}{,}\PY{l+m+mi}{5}\PY{p}{)}\PY{p}{)}
    
    \PY{n}{plt}\PY{o}{.}\PY{n}{bar}\PY{p}{(}\PY{n}{x}\PY{p}{,} \PY{n}{y}\PY{p}{)}

    \PY{c+c1}{\PYZsh{} Labeling the axes}
    \PY{n}{plt}\PY{o}{.}\PY{n}{xlabel}\PY{p}{(}\PY{n}{col\PYZus{}name}\PY{p}{)}
    \PY{n}{plt}\PY{o}{.}\PY{n}{ylabel}\PY{p}{(}\PY{l+s+s1}{\PYZsq{}}\PY{l+s+s1}{count}\PY{l+s+s1}{\PYZsq{}}\PY{p}{)}
    
    \PY{c+c1}{\PYZsh{} Dsiplay the plot}
    \PY{n}{plt}\PY{o}{.}\PY{n}{show}\PY{p}{(}\PY{p}{)}
    \PY{c+c1}{\PYZsh{}return df\PYZus{}go\PYZus{}bike.col\PYZus{}name.unique() , df\PYZus{}go\PYZus{}bike.col\PYZus{}name.value\PYZus{}counts()}
\end{Verbatim}
\end{tcolorbox}

    \begin{tcolorbox}[breakable, size=fbox, boxrule=1pt, pad at break*=1mm,colback=cellbackground, colframe=cellborder]
\prompt{In}{incolor}{62}{\boxspacing}
\begin{Verbatim}[commandchars=\\\{\}]
\PY{k}{def} \PY{n+nf}{pie\PYZus{}chart} \PY{p}{(}\PY{n}{col\PYZus{}name}\PY{p}{)}\PY{p}{:}     
    \PY{n}{sorted\PYZus{}counts} \PY{o}{=} \PY{n}{df\PYZus{}go\PYZus{}bike}\PY{p}{[}\PY{n}{col\PYZus{}name}\PY{p}{]}\PY{o}{.}\PY{n}{value\PYZus{}counts}\PY{p}{(}\PY{p}{)}
    \PY{n}{plt}\PY{o}{.}\PY{n}{figure}\PY{p}{(}\PY{n}{figsize}\PY{o}{=}\PY{p}{(}\PY{l+m+mi}{10}\PY{p}{,}\PY{l+m+mi}{5}\PY{p}{)}\PY{p}{)}
    
    \PY{n}{plt}\PY{o}{.}\PY{n}{pie}\PY{p}{(}\PY{n}{sorted\PYZus{}counts}\PY{p}{,} \PY{n}{labels} \PY{o}{=} \PY{n}{sorted\PYZus{}counts}\PY{o}{.}\PY{n}{index}\PY{p}{,} \PY{n}{startangle} \PY{o}{=} \PY{l+m+mi}{150}\PY{p}{,}
            \PY{n}{counterclock} \PY{o}{=} \PY{k+kc}{False}\PY{p}{,} \PY{n}{wedgeprops} \PY{o}{=} \PY{p}{\PYZob{}}\PY{l+s+s1}{\PYZsq{}}\PY{l+s+s1}{width}\PY{l+s+s1}{\PYZsq{}} \PY{p}{:} \PY{l+m+mf}{0.4}\PY{p}{\PYZcb{}}\PY{p}{)}\PY{p}{;}
    \PY{n}{plt}\PY{o}{.}\PY{n}{axis}\PY{p}{(}\PY{l+s+s1}{\PYZsq{}}\PY{l+s+s1}{square}\PY{l+s+s1}{\PYZsq{}}\PY{p}{)}
\end{Verbatim}
\end{tcolorbox}

    \begin{tcolorbox}[breakable, size=fbox, boxrule=1pt, pad at break*=1mm,colback=cellbackground, colframe=cellborder]
\prompt{In}{incolor}{63}{\boxspacing}
\begin{Verbatim}[commandchars=\\\{\}]
\PY{n}{df\PYZus{}go\PYZus{}bike}\PY{o}{.}\PY{n}{duration\PYZus{}sec}\PY{o}{.}\PY{n}{unique}\PY{p}{(}\PY{p}{)}
\end{Verbatim}
\end{tcolorbox}

            \begin{tcolorbox}[breakable, size=fbox, boxrule=.5pt, pad at break*=1mm, opacityfill=0]
\prompt{Out}{outcolor}{63}{\boxspacing}
\begin{Verbatim}[commandchars=\\\{\}]
array([52185, 61854,  1585, {\ldots},  2780,  5713,  2822], dtype=int64)
\end{Verbatim}
\end{tcolorbox}
        
    \begin{tcolorbox}[breakable, size=fbox, boxrule=1pt, pad at break*=1mm,colback=cellbackground, colframe=cellborder]
\prompt{In}{incolor}{64}{\boxspacing}
\begin{Verbatim}[commandchars=\\\{\}]
\PY{n}{df\PYZus{}go\PYZus{}bike}\PY{o}{.}\PY{n}{duration\PYZus{}sec}\PY{o}{.}\PY{n}{value\PYZus{}counts}\PY{p}{(}\PY{p}{)}
\end{Verbatim}
\end{tcolorbox}

            \begin{tcolorbox}[breakable, size=fbox, boxrule=.5pt, pad at break*=1mm, opacityfill=0]
\prompt{Out}{outcolor}{64}{\boxspacing}
\begin{Verbatim}[commandchars=\\\{\}]
272     303
323     278
305     277
369     275
324     272
       {\ldots}
9622      1
3263      1
3407      1
6141      1
2822      1
Name: duration\_sec, Length: 4341, dtype: int64
\end{Verbatim}
\end{tcolorbox}
        
    \hypertarget{distribution-of-duration-trip}{%
\subsection{Distribution of duration
trip}\label{distribution-of-duration-trip}}

duration\_sec has a long-tailed distribution the duration\_sec
distribution looks roughly bimodal, with one peak between 250 and 300.
Interestingly, there's a steep jump in frequency right before 200,
rather than a smooth ramp up.

    \begin{tcolorbox}[breakable, size=fbox, boxrule=1pt, pad at break*=1mm,colback=cellbackground, colframe=cellborder]
\prompt{In}{incolor}{65}{\boxspacing}
\begin{Verbatim}[commandchars=\\\{\}]
\PY{n}{bar\PYZus{}plot}\PY{p}{(}\PY{l+s+s1}{\PYZsq{}}\PY{l+s+s1}{duration\PYZus{}sec}\PY{l+s+s1}{\PYZsq{}}\PY{p}{)}
\end{Verbatim}
\end{tcolorbox}

    \begin{center}
    \adjustimage{max size={0.9\linewidth}{0.9\paperheight}}{Ford_Go_Bike_Part1_files/Ford_Go_Bike_Part1_102_0.png}
    \end{center}
    { \hspace*{\fill} \\}
    
    duration\_sec has a long-tailed distribution the duration\_sec
distribution looks roughly bimodal, with one peak between 250 and 300.
Interestingly, there's a steep jump in frequency right before 200,
rather than a smooth ramp up.

    \begin{tcolorbox}[breakable, size=fbox, boxrule=1pt, pad at break*=1mm,colback=cellbackground, colframe=cellborder]
\prompt{In}{incolor}{66}{\boxspacing}
\begin{Verbatim}[commandchars=\\\{\}]
\PY{n}{bar\PYZus{}plot}\PY{p}{(}\PY{l+s+s1}{\PYZsq{}}\PY{l+s+s1}{duration\PYZus{}minu}\PY{l+s+s1}{\PYZsq{}}\PY{p}{)}
\end{Verbatim}
\end{tcolorbox}

    \begin{center}
    \adjustimage{max size={0.9\linewidth}{0.9\paperheight}}{Ford_Go_Bike_Part1_files/Ford_Go_Bike_Part1_104_0.png}
    \end{center}
    { \hspace*{\fill} \\}
    
    duration\_minu has a long-tailed distribution the duration\_minu
distribution looks roughly bimodal, with one peak between more 300.
Interestingly, there's a steep jump in frequency right before 200,
rather than a smooth ramp up.

    \begin{tcolorbox}[breakable, size=fbox, boxrule=1pt, pad at break*=1mm,colback=cellbackground, colframe=cellborder]
\prompt{In}{incolor}{67}{\boxspacing}
\begin{Verbatim}[commandchars=\\\{\}]
\PY{n}{bar\PYZus{}plot}\PY{p}{(}\PY{l+s+s1}{\PYZsq{}}\PY{l+s+s1}{duration\PYZus{}hr}\PY{l+s+s1}{\PYZsq{}}\PY{p}{)}
\end{Verbatim}
\end{tcolorbox}

    \begin{center}
    \adjustimage{max size={0.9\linewidth}{0.9\paperheight}}{Ford_Go_Bike_Part1_files/Ford_Go_Bike_Part1_106_0.png}
    \end{center}
    { \hspace*{\fill} \\}
    
    duration\_hr has a long-tailed distribution the duration\_hr
distribution looks roughly bimodal, with one peak between more 300.
Interestingly, there's a steep jump in frequency right before 200,
rather than a smooth ramp up.

    \begin{tcolorbox}[breakable, size=fbox, boxrule=1pt, pad at break*=1mm,colback=cellbackground, colframe=cellborder]
\prompt{In}{incolor}{68}{\boxspacing}
\begin{Verbatim}[commandchars=\\\{\}]
\PY{n}{bar\PYZus{}plot}\PY{p}{(}\PY{l+s+s1}{\PYZsq{}}\PY{l+s+s1}{duration\PYZus{}days}\PY{l+s+s1}{\PYZsq{}}\PY{p}{)}
\end{Verbatim}
\end{tcolorbox}

    \begin{center}
    \adjustimage{max size={0.9\linewidth}{0.9\paperheight}}{Ford_Go_Bike_Part1_files/Ford_Go_Bike_Part1_108_0.png}
    \end{center}
    { \hspace*{\fill} \\}
    
    duration\_days has a long-tailed distribution the duration\_days
distribution looks roughly bimodal, with one peak between more 300.
Interestingly, there's a steep jump in frequency right before 200,
rather than a smooth ramp up.

    \begin{tcolorbox}[breakable, size=fbox, boxrule=1pt, pad at break*=1mm,colback=cellbackground, colframe=cellborder]
\prompt{In}{incolor}{69}{\boxspacing}
\begin{Verbatim}[commandchars=\\\{\}]
\PY{n}{df\PYZus{}go\PYZus{}bike}\PY{o}{.}\PY{n}{member\PYZus{}gender}\PY{o}{.}\PY{n}{unique}\PY{p}{(}\PY{p}{)}
\end{Verbatim}
\end{tcolorbox}

            \begin{tcolorbox}[breakable, size=fbox, boxrule=.5pt, pad at break*=1mm, opacityfill=0]
\prompt{Out}{outcolor}{69}{\boxspacing}
\begin{Verbatim}[commandchars=\\\{\}]
array(['Male', 'Female'], dtype=object)
\end{Verbatim}
\end{tcolorbox}
        
    \begin{tcolorbox}[breakable, size=fbox, boxrule=1pt, pad at break*=1mm,colback=cellbackground, colframe=cellborder]
\prompt{In}{incolor}{70}{\boxspacing}
\begin{Verbatim}[commandchars=\\\{\}]
\PY{n}{df\PYZus{}go\PYZus{}bike}\PY{o}{.}\PY{n}{member\PYZus{}gender}\PY{o}{.}\PY{n}{value\PYZus{}counts}\PY{p}{(}\PY{p}{)}
\end{Verbatim}
\end{tcolorbox}

            \begin{tcolorbox}[breakable, size=fbox, boxrule=.5pt, pad at break*=1mm, opacityfill=0]
\prompt{Out}{outcolor}{70}{\boxspacing}
\begin{Verbatim}[commandchars=\\\{\}]
Male      130500
Female     40805
Name: member\_gender, dtype: int64
\end{Verbatim}
\end{tcolorbox}
        
    \hypertarget{distribution-of-member_gender}{%
\subsection{Distribution of
member\_gender}\label{distribution-of-member_gender}}

in the figure , there are bar plot / pie\_chart to showup and compare
between in number of users whether male or female

    \begin{tcolorbox}[breakable, size=fbox, boxrule=1pt, pad at break*=1mm,colback=cellbackground, colframe=cellborder]
\prompt{In}{incolor}{71}{\boxspacing}
\begin{Verbatim}[commandchars=\\\{\}]
\PY{n}{bar\PYZus{}plot}\PY{p}{(}\PY{l+s+s1}{\PYZsq{}}\PY{l+s+s1}{member\PYZus{}gender}\PY{l+s+s1}{\PYZsq{}}\PY{p}{)}
\end{Verbatim}
\end{tcolorbox}

    \begin{center}
    \adjustimage{max size={0.9\linewidth}{0.9\paperheight}}{Ford_Go_Bike_Part1_files/Ford_Go_Bike_Part1_113_0.png}
    \end{center}
    { \hspace*{\fill} \\}
    
    in the figure , there are bar plot to showup and compare between in
number of users whether male or female

    \begin{tcolorbox}[breakable, size=fbox, boxrule=1pt, pad at break*=1mm,colback=cellbackground, colframe=cellborder]
\prompt{In}{incolor}{72}{\boxspacing}
\begin{Verbatim}[commandchars=\\\{\}]
\PY{n}{pie\PYZus{}chart}\PY{p}{(}\PY{l+s+s1}{\PYZsq{}}\PY{l+s+s1}{member\PYZus{}gender}\PY{l+s+s1}{\PYZsq{}}\PY{p}{)}
\end{Verbatim}
\end{tcolorbox}

    \begin{center}
    \adjustimage{max size={0.9\linewidth}{0.9\paperheight}}{Ford_Go_Bike_Part1_files/Ford_Go_Bike_Part1_115_0.png}
    \end{center}
    { \hspace*{\fill} \\}
    
    in the figure , there are pie chart to showup and compare between in
number of users whether male or female

    \begin{tcolorbox}[breakable, size=fbox, boxrule=1pt, pad at break*=1mm,colback=cellbackground, colframe=cellborder]
\prompt{In}{incolor}{73}{\boxspacing}
\begin{Verbatim}[commandchars=\\\{\}]
\PY{n}{df\PYZus{}go\PYZus{}bike}\PY{o}{.}\PY{n}{user\PYZus{}type}\PY{o}{.}\PY{n}{unique}\PY{p}{(}\PY{p}{)}
\end{Verbatim}
\end{tcolorbox}

            \begin{tcolorbox}[breakable, size=fbox, boxrule=.5pt, pad at break*=1mm, opacityfill=0]
\prompt{Out}{outcolor}{73}{\boxspacing}
\begin{Verbatim}[commandchars=\\\{\}]
array(['Customer', 'Subscriber'], dtype=object)
\end{Verbatim}
\end{tcolorbox}
        
    \begin{tcolorbox}[breakable, size=fbox, boxrule=1pt, pad at break*=1mm,colback=cellbackground, colframe=cellborder]
\prompt{In}{incolor}{74}{\boxspacing}
\begin{Verbatim}[commandchars=\\\{\}]
\PY{n}{df\PYZus{}go\PYZus{}bike}\PY{o}{.}\PY{n}{user\PYZus{}type}\PY{o}{.}\PY{n}{value\PYZus{}counts}\PY{p}{(}\PY{p}{)}
\end{Verbatim}
\end{tcolorbox}

            \begin{tcolorbox}[breakable, size=fbox, boxrule=.5pt, pad at break*=1mm, opacityfill=0]
\prompt{Out}{outcolor}{74}{\boxspacing}
\begin{Verbatim}[commandchars=\\\{\}]
Subscriber    155189
Customer       16116
Name: user\_type, dtype: int64
\end{Verbatim}
\end{tcolorbox}
        
    \hypertarget{distribution-of-user_type}{%
\subsection{Distribution of
user\_type}\label{distribution-of-user_type}}

in the figure , there are bar plot / pie\_chart to showup and compare
between in user\_type whether customer or subscribe

    \begin{tcolorbox}[breakable, size=fbox, boxrule=1pt, pad at break*=1mm,colback=cellbackground, colframe=cellborder]
\prompt{In}{incolor}{75}{\boxspacing}
\begin{Verbatim}[commandchars=\\\{\}]
\PY{n}{bar\PYZus{}plot}\PY{p}{(}\PY{l+s+s1}{\PYZsq{}}\PY{l+s+s1}{user\PYZus{}type}\PY{l+s+s1}{\PYZsq{}}\PY{p}{)}
\end{Verbatim}
\end{tcolorbox}

    \begin{center}
    \adjustimage{max size={0.9\linewidth}{0.9\paperheight}}{Ford_Go_Bike_Part1_files/Ford_Go_Bike_Part1_120_0.png}
    \end{center}
    { \hspace*{\fill} \\}
    
    in the figure , there are bar plot to showup and compare between in
number of users whether customer or subscriber

    \begin{tcolorbox}[breakable, size=fbox, boxrule=1pt, pad at break*=1mm,colback=cellbackground, colframe=cellborder]
\prompt{In}{incolor}{76}{\boxspacing}
\begin{Verbatim}[commandchars=\\\{\}]
\PY{n}{pie\PYZus{}chart}\PY{p}{(}\PY{l+s+s1}{\PYZsq{}}\PY{l+s+s1}{user\PYZus{}type}\PY{l+s+s1}{\PYZsq{}}\PY{p}{)}
\end{Verbatim}
\end{tcolorbox}

    \begin{center}
    \adjustimage{max size={0.9\linewidth}{0.9\paperheight}}{Ford_Go_Bike_Part1_files/Ford_Go_Bike_Part1_122_0.png}
    \end{center}
    { \hspace*{\fill} \\}
    
    in the figure , there are pie chart to showup and compare between in
number of users whether customer or subscriber

    \begin{tcolorbox}[breakable, size=fbox, boxrule=1pt, pad at break*=1mm,colback=cellbackground, colframe=cellborder]
\prompt{In}{incolor}{77}{\boxspacing}
\begin{Verbatim}[commandchars=\\\{\}]
\PY{n}{df\PYZus{}go\PYZus{}bike}\PY{o}{.}\PY{n}{info}\PY{p}{(}\PY{p}{)}
\end{Verbatim}
\end{tcolorbox}

    \begin{Verbatim}[commandchars=\\\{\}]
<class 'pandas.core.frame.DataFrame'>
RangeIndex: 171305 entries, 0 to 171304
Data columns (total 22 columns):
 \#   Column                   Non-Null Count   Dtype
---  ------                   --------------   -----
 0   Unnamed: 0               171305 non-null  int64
 1   duration\_sec             171305 non-null  int64
 2   start\_time               171305 non-null  object
 3   end\_time                 171305 non-null  object
 4   start\_station\_id         171305 non-null  float64
 5   start\_station\_name       171305 non-null  object
 6   start\_station\_latitude   171305 non-null  float64
 7   start\_station\_longitude  171305 non-null  float64
 8   end\_station\_id           171305 non-null  float64
 9   end\_station\_name         171305 non-null  object
 10  end\_station\_latitude     171305 non-null  float64
 11  end\_station\_longitude    171305 non-null  float64
 12  bike\_id                  171305 non-null  int64
 13  user\_type                171305 non-null  object
 14  member\_birth\_year        171305 non-null  float64
 15  member\_gender            171305 non-null  object
 16  bike\_share\_for\_all\_trip  171305 non-null  object
 17  duration\_minu            171305 non-null  float64
 18  duration\_hr              171305 non-null  float64
 19  duration\_days            171305 non-null  float64
 20  duration\_weeks           171305 non-null  float64
 21  duration\_months          171305 non-null  float64
dtypes: float64(12), int64(3), object(7)
memory usage: 28.8+ MB
    \end{Verbatim}

    \begin{tcolorbox}[breakable, size=fbox, boxrule=1pt, pad at break*=1mm,colback=cellbackground, colframe=cellborder]
\prompt{In}{incolor}{78}{\boxspacing}
\begin{Verbatim}[commandchars=\\\{\}]
\PY{n}{df\PYZus{}go\PYZus{}bike}\PY{o}{.}\PY{n}{member\PYZus{}birth\PYZus{}year}\PY{o}{.}\PY{n}{value\PYZus{}counts}\PY{p}{(}\PY{p}{)}
\end{Verbatim}
\end{tcolorbox}

            \begin{tcolorbox}[breakable, size=fbox, boxrule=.5pt, pad at break*=1mm, opacityfill=0]
\prompt{Out}{outcolor}{78}{\boxspacing}
\begin{Verbatim}[commandchars=\\\{\}]
1988.0    10015
1993.0     9145
1989.0     8805
1990.0     8495
1991.0     8339
          {\ldots}
1938.0        3
1944.0        2
1934.0        2
1878.0        1
1927.0        1
Name: member\_birth\_year, Length: 72, dtype: int64
\end{Verbatim}
\end{tcolorbox}
        
    \begin{tcolorbox}[breakable, size=fbox, boxrule=1pt, pad at break*=1mm,colback=cellbackground, colframe=cellborder]
\prompt{In}{incolor}{79}{\boxspacing}
\begin{Verbatim}[commandchars=\\\{\}]
\PY{n}{df\PYZus{}go\PYZus{}bike}\PY{o}{.}\PY{n}{member\PYZus{}birth\PYZus{}year}\PY{o}{.}\PY{n}{unique}\PY{p}{(}\PY{p}{)}
\end{Verbatim}
\end{tcolorbox}

            \begin{tcolorbox}[breakable, size=fbox, boxrule=.5pt, pad at break*=1mm, opacityfill=0]
\prompt{Out}{outcolor}{79}{\boxspacing}
\begin{Verbatim}[commandchars=\\\{\}]
array([1984., 1972., 1974., 1959., 1983., 1989., 1992., 1996., 1993.,
       1990., 1988., 1981., 1975., 1978., 1991., 1997., 1986., 2000.,
       1982., 1995., 1980., 1973., 1985., 1971., 1979., 1967., 1998.,
       1994., 1977., 1999., 1987., 1969., 1963., 1976., 1964., 1965.,
       1961., 1968., 1966., 1962., 1954., 1958., 1960., 1970., 1956.,
       1957., 1945., 1900., 1952., 1948., 1951., 1941., 1950., 1949.,
       1953., 1955., 1946., 1947., 1931., 1943., 1942., 1920., 1933.,
       2001., 1878., 1901., 1944., 1934., 1939., 1902., 1938., 1927.])
\end{Verbatim}
\end{tcolorbox}
        
    \hypertarget{distribution-of-member_birth_year}{%
\subsection{Distribution of
member\_birth\_year}\label{distribution-of-member_birth_year}}

in the figure , there are bar plot / pie\_chart to showup in
member\_birth\_year

    \begin{tcolorbox}[breakable, size=fbox, boxrule=1pt, pad at break*=1mm,colback=cellbackground, colframe=cellborder]
\prompt{In}{incolor}{80}{\boxspacing}
\begin{Verbatim}[commandchars=\\\{\}]
\PY{n}{bar\PYZus{}plot}\PY{p}{(}\PY{l+s+s1}{\PYZsq{}}\PY{l+s+s1}{member\PYZus{}birth\PYZus{}year}\PY{l+s+s1}{\PYZsq{}}\PY{p}{)}
\end{Verbatim}
\end{tcolorbox}

    \begin{center}
    \adjustimage{max size={0.9\linewidth}{0.9\paperheight}}{Ford_Go_Bike_Part1_files/Ford_Go_Bike_Part1_128_0.png}
    \end{center}
    { \hspace*{\fill} \\}
    
    member\_birth\_year has a long-tailed distribution the
member\_birth\_year distribution looks roughly bimodal, with one peak
between 8000 and 10000. Interestingly, there's a steep jump in frequency
right before 6000, rather than a smooth ramp up.

    \begin{tcolorbox}[breakable, size=fbox, boxrule=1pt, pad at break*=1mm,colback=cellbackground, colframe=cellborder]
\prompt{In}{incolor}{81}{\boxspacing}
\begin{Verbatim}[commandchars=\\\{\}]
\PY{n}{df\PYZus{}go\PYZus{}bike}\PY{o}{.}\PY{n}{info}\PY{p}{(}\PY{p}{)}
\end{Verbatim}
\end{tcolorbox}

    \begin{Verbatim}[commandchars=\\\{\}]
<class 'pandas.core.frame.DataFrame'>
RangeIndex: 171305 entries, 0 to 171304
Data columns (total 22 columns):
 \#   Column                   Non-Null Count   Dtype
---  ------                   --------------   -----
 0   Unnamed: 0               171305 non-null  int64
 1   duration\_sec             171305 non-null  int64
 2   start\_time               171305 non-null  object
 3   end\_time                 171305 non-null  object
 4   start\_station\_id         171305 non-null  float64
 5   start\_station\_name       171305 non-null  object
 6   start\_station\_latitude   171305 non-null  float64
 7   start\_station\_longitude  171305 non-null  float64
 8   end\_station\_id           171305 non-null  float64
 9   end\_station\_name         171305 non-null  object
 10  end\_station\_latitude     171305 non-null  float64
 11  end\_station\_longitude    171305 non-null  float64
 12  bike\_id                  171305 non-null  int64
 13  user\_type                171305 non-null  object
 14  member\_birth\_year        171305 non-null  float64
 15  member\_gender            171305 non-null  object
 16  bike\_share\_for\_all\_trip  171305 non-null  object
 17  duration\_minu            171305 non-null  float64
 18  duration\_hr              171305 non-null  float64
 19  duration\_days            171305 non-null  float64
 20  duration\_weeks           171305 non-null  float64
 21  duration\_months          171305 non-null  float64
dtypes: float64(12), int64(3), object(7)
memory usage: 28.8+ MB
    \end{Verbatim}

    \hypertarget{discuss-the-distributions-of-your-variables-of-interest.-were-there-any-unusual-points-did-you-need-to-perform-any-transformations}{%
\subsubsection{Discuss the distribution(s) of your variable(s) of
interest. Were there any unusual points? Did you need to perform any
transformations?}\label{discuss-the-distributions-of-your-variables-of-interest.-were-there-any-unusual-points-did-you-need-to-perform-any-transformations}}

\hypertarget{of-the-features-you-investigated-were-there-any-unusual-distributions-did-you-perform-any-operations-on-the-data-to-tidy-adjust-or-change-the-form-of-the-data-if-so-why-did-you-do-this}{%
\subsubsection{Of the features you investigated, were there any unusual
distributions? Did you perform any operations on the data to tidy,
adjust, or change the form of the data? If so, why did you do
this?}\label{of-the-features-you-investigated-were-there-any-unusual-distributions-did-you-perform-any-operations-on-the-data-to-tidy-adjust-or-change-the-form-of-the-data-if-so-why-did-you-do-this}}

When investigating the x, y, and z size variables, a number of outlier
points were identified. Overall, these points can be characterized by an
inconsistency between the recorded value of depth, and the value that
would be derived from using x, y, and z. For safety, all of these points
were removed from the dataset to move forwards.

\hypertarget{bivariate-exploration}{%
\subsection{Bivariate Exploration}\label{bivariate-exploration}}

To start off with, I want to look at the pairwise correlations present
between features in the data.

    \begin{tcolorbox}[breakable, size=fbox, boxrule=1pt, pad at break*=1mm,colback=cellbackground, colframe=cellborder]
\prompt{In}{incolor}{82}{\boxspacing}
\begin{Verbatim}[commandchars=\\\{\}]
\PY{k}{def} \PY{n+nf}{bivar} \PY{p}{(}\PY{n}{x1} \PY{p}{,} \PY{n}{y1} \PY{p}{,} \PY{n}{x2} \PY{p}{,} \PY{n}{y2}\PY{p}{)}\PY{p}{:}
    \PY{n}{plt}\PY{o}{.}\PY{n}{figure}\PY{p}{(}\PY{n}{figsize} \PY{o}{=} \PY{p}{[}\PY{l+m+mi}{18}\PY{p}{,} \PY{l+m+mi}{6}\PY{p}{]}\PY{p}{)}

    \PY{c+c1}{\PYZsh{} PLOT ON LEFT}
    \PY{n}{plt}\PY{o}{.}\PY{n}{subplot}\PY{p}{(}\PY{l+m+mi}{1}\PY{p}{,} \PY{l+m+mi}{2}\PY{p}{,} \PY{l+m+mi}{1}\PY{p}{)}
    \PY{n}{sns}\PY{o}{.}\PY{n}{regplot}\PY{p}{(}\PY{n}{data} \PY{o}{=} \PY{n}{df\PYZus{}go\PYZus{}bike} \PY{p}{,} \PY{n}{x} \PY{o}{=} \PY{n}{x1} \PY{p}{,} \PY{n}{y} \PY{o}{=} \PY{n}{y1} \PY{p}{,} \PY{n}{x\PYZus{}jitter}\PY{o}{=}\PY{l+m+mf}{0.04}\PY{p}{,} \PY{n}{scatter\PYZus{}kws}\PY{o}{=}\PY{p}{\PYZob{}}\PY{l+s+s1}{\PYZsq{}}\PY{l+s+s1}{alpha}\PY{l+s+s1}{\PYZsq{}}\PY{p}{:}\PY{l+m+mi}{1}\PY{o}{/}\PY{l+m+mi}{10}\PY{p}{\PYZcb{}}\PY{p}{,} \PY{n}{fit\PYZus{}reg}\PY{o}{=}\PY{k+kc}{False}\PY{p}{)}
    \PY{n}{plt}\PY{o}{.}\PY{n}{xlabel}\PY{p}{(}\PY{n}{x1}\PY{p}{)}
    \PY{n}{plt}\PY{o}{.}\PY{n}{ylabel}\PY{p}{(}\PY{n}{y1}\PY{p}{)}\PY{p}{;}

    \PY{c+c1}{\PYZsh{} PLOT ON RIGHT}
    \PY{n}{plt}\PY{o}{.}\PY{n}{subplot}\PY{p}{(}\PY{l+m+mi}{1}\PY{p}{,} \PY{l+m+mi}{2}\PY{p}{,} \PY{l+m+mi}{2}\PY{p}{)}
    \PY{n}{plt}\PY{o}{.}\PY{n}{hist2d}\PY{p}{(}\PY{n}{data} \PY{o}{=} \PY{n}{df\PYZus{}go\PYZus{}bike} \PY{p}{,} \PY{n}{x} \PY{o}{=} \PY{n}{x2} \PY{p}{,} \PY{n}{y} \PY{o}{=} \PY{n}{y2}\PY{p}{)}
    \PY{n}{plt}\PY{o}{.}\PY{n}{colorbar}\PY{p}{(}\PY{p}{)}
    \PY{n}{plt}\PY{o}{.}\PY{n}{xlabel}\PY{p}{(}\PY{n}{x2}\PY{p}{)}
    \PY{n}{plt}\PY{o}{.}\PY{n}{ylabel}\PY{p}{(}\PY{n}{y2}\PY{p}{)}\PY{p}{;}
\end{Verbatim}
\end{tcolorbox}

    \begin{tcolorbox}[breakable, size=fbox, boxrule=1pt, pad at break*=1mm,colback=cellbackground, colframe=cellborder]
\prompt{In}{incolor}{83}{\boxspacing}
\begin{Verbatim}[commandchars=\\\{\}]
\PY{n}{bivar} \PY{p}{(}\PY{l+s+s1}{\PYZsq{}}\PY{l+s+s1}{member\PYZus{}birth\PYZus{}year}\PY{l+s+s1}{\PYZsq{}}\PY{p}{,}\PY{l+s+s1}{\PYZsq{}}\PY{l+s+s1}{member\PYZus{}gender}\PY{l+s+s1}{\PYZsq{}}\PY{p}{,}\PY{l+s+s1}{\PYZsq{}}\PY{l+s+s1}{member\PYZus{}birth\PYZus{}year}\PY{l+s+s1}{\PYZsq{}}\PY{p}{,}\PY{l+s+s1}{\PYZsq{}}\PY{l+s+s1}{duration\PYZus{}hr}\PY{l+s+s1}{\PYZsq{}}\PY{p}{)}
\end{Verbatim}
\end{tcolorbox}

    \begin{center}
    \adjustimage{max size={0.9\linewidth}{0.9\paperheight}}{Ford_Go_Bike_Part1_files/Ford_Go_Bike_Part1_133_0.png}
    \end{center}
    { \hspace*{\fill} \\}
    
    \begin{tcolorbox}[breakable, size=fbox, boxrule=1pt, pad at break*=1mm,colback=cellbackground, colframe=cellborder]
\prompt{In}{incolor}{84}{\boxspacing}
\begin{Verbatim}[commandchars=\\\{\}]
\PY{k}{def} \PY{n+nf}{subplots}\PY{p}{(}\PY{n}{x}\PY{p}{,}\PY{n}{y}\PY{p}{)}\PY{p}{:}
    \PY{c+c1}{\PYZsh{} since there\PYZsq{}s only three subplots to create, using the full data should be fine.}
    \PY{n}{plt}\PY{o}{.}\PY{n}{figure}\PY{p}{(}\PY{n}{figsize} \PY{o}{=} \PY{p}{[}\PY{l+m+mi}{35} \PY{p}{,} \PY{l+m+mi}{35}\PY{p}{]}\PY{p}{)}

    \PY{c+c1}{\PYZsh{} subplot 1: color vs cut}
    \PY{n}{ax} \PY{o}{=} \PY{n}{plt}\PY{o}{.}\PY{n}{subplot}\PY{p}{(}\PY{l+m+mi}{3}\PY{p}{,} \PY{l+m+mi}{1}\PY{p}{,} \PY{l+m+mi}{1}\PY{p}{)}
    \PY{n}{sns}\PY{o}{.}\PY{n}{countplot}\PY{p}{(}\PY{n}{data} \PY{o}{=} \PY{n}{df\PYZus{}go\PYZus{}bike}\PY{p}{,} \PY{n}{x} \PY{o}{=} \PY{n}{x}\PY{p}{,} \PY{n}{hue} \PY{o}{=} \PY{n}{y}\PY{p}{,} \PY{n}{palette} \PY{o}{=} \PY{l+s+s1}{\PYZsq{}}\PY{l+s+s1}{Blues}\PY{l+s+s1}{\PYZsq{}}\PY{p}{)}
    \PY{n}{ax}\PY{o}{.}\PY{n}{legend}\PY{p}{(}\PY{n}{ncol} \PY{o}{=} \PY{l+m+mi}{2}\PY{p}{)} \PY{c+c1}{\PYZsh{} re\PYZhy{}arrange legend to reduce overlapping}
\end{Verbatim}
\end{tcolorbox}

    \hypertarget{member_birth_year-vs-member_gender}{%
\subsection{member\_birth\_year vs
member\_gender}\label{member_birth_year-vs-member_gender}}

Plotting member\_birth\_year linear relationship. For member\_gender ,
there appears to be a member\_birth\_year: based on the trend
member\_gender , we might expect member\_birth\_year to take
member\_gender.

    \begin{tcolorbox}[breakable, size=fbox, boxrule=1pt, pad at break*=1mm,colback=cellbackground, colframe=cellborder]
\prompt{In}{incolor}{85}{\boxspacing}
\begin{Verbatim}[commandchars=\\\{\}]
\PY{n}{subplots}\PY{p}{(}\PY{l+s+s1}{\PYZsq{}}\PY{l+s+s1}{member\PYZus{}gender}\PY{l+s+s1}{\PYZsq{}}\PY{p}{,}\PY{l+s+s1}{\PYZsq{}}\PY{l+s+s1}{member\PYZus{}birth\PYZus{}year}\PY{l+s+s1}{\PYZsq{}}\PY{p}{)}
\end{Verbatim}
\end{tcolorbox}

    \begin{center}
    \adjustimage{max size={0.9\linewidth}{0.9\paperheight}}{Ford_Go_Bike_Part1_files/Ford_Go_Bike_Part1_136_0.png}
    \end{center}
    { \hspace*{\fill} \\}
    
    \begin{tcolorbox}[breakable, size=fbox, boxrule=1pt, pad at break*=1mm,colback=cellbackground, colframe=cellborder]
\prompt{In}{incolor}{86}{\boxspacing}
\begin{Verbatim}[commandchars=\\\{\}]
\PY{k}{def} \PY{n+nf}{scatterplots}\PY{p}{(}\PY{n}{x}\PY{p}{,}\PY{n}{y}\PY{p}{)}\PY{p}{:}    
    \PY{c+c1}{\PYZsh{} scatter plot of price vs. carat, with log transform on price axis}

    \PY{n}{plt}\PY{o}{.}\PY{n}{figure}\PY{p}{(}\PY{n}{figsize} \PY{o}{=} \PY{p}{[}\PY{l+m+mi}{8}\PY{p}{,} \PY{l+m+mi}{6}\PY{p}{]}\PY{p}{)}
    \PY{n}{plt}\PY{o}{.}\PY{n}{scatter}\PY{p}{(}\PY{n}{data} \PY{o}{=} \PY{n}{df\PYZus{}go\PYZus{}bike}\PY{p}{,} \PY{n}{x} \PY{o}{=} \PY{n}{x}\PY{p}{,} \PY{n}{y} \PY{o}{=} \PY{n}{y}\PY{p}{,} \PY{n}{alpha} \PY{o}{=} \PY{l+m+mi}{1}\PY{o}{/}\PY{l+m+mi}{10}\PY{p}{)}
    \PY{n}{plt}\PY{o}{.}\PY{n}{xlabel}\PY{p}{(}\PY{n}{x}\PY{p}{)}
    \PY{n}{plt}\PY{o}{.}\PY{n}{ylabel}\PY{p}{(}\PY{n}{y}\PY{p}{)}
    \PY{n}{plt}\PY{o}{.}\PY{n}{show}\PY{p}{(}\PY{p}{)}
\end{Verbatim}
\end{tcolorbox}

    using plotting.scatter\_matrix() function to draw Histogramsfor each
column and scatter plotting between numerical columns and use figsize(,)
parameter to show it obviously

    \hypertarget{member_birth_year-vs-duration_hr}{%
\subsection{member\_birth\_year vs
duration\_hr}\label{member_birth_year-vs-duration_hr}}

Plotting member\_birth\_year linear relationship. For duration\_hr above
0.5h , there appears to be a member\_birth\_year: based on the trend
below 20h duration\_hr , we might expect member\_birth\_year to take
duration\_hr between 0 and 5

    \begin{tcolorbox}[breakable, size=fbox, boxrule=1pt, pad at break*=1mm,colback=cellbackground, colframe=cellborder]
\prompt{In}{incolor}{87}{\boxspacing}
\begin{Verbatim}[commandchars=\\\{\}]
\PY{n}{scatterplots}\PY{p}{(}\PY{l+s+s1}{\PYZsq{}}\PY{l+s+s1}{member\PYZus{}birth\PYZus{}year}\PY{l+s+s1}{\PYZsq{}}\PY{p}{,}\PY{l+s+s1}{\PYZsq{}}\PY{l+s+s1}{duration\PYZus{}hr}\PY{l+s+s1}{\PYZsq{}}\PY{p}{)}
\end{Verbatim}
\end{tcolorbox}

    \begin{center}
    \adjustimage{max size={0.9\linewidth}{0.9\paperheight}}{Ford_Go_Bike_Part1_files/Ford_Go_Bike_Part1_140_0.png}
    \end{center}
    { \hspace*{\fill} \\}
    
    \begin{tcolorbox}[breakable, size=fbox, boxrule=1pt, pad at break*=1mm,colback=cellbackground, colframe=cellborder]
\prompt{In}{incolor}{ }{\boxspacing}
\begin{Verbatim}[commandchars=\\\{\}]

\end{Verbatim}
\end{tcolorbox}

    \hypertarget{talk-about-some-of-the-relationships-you-observed-in-this-part-of-the-investigation.-how-did-the-features-of-interest-vary-with-other-features-in-the-dataset}{%
\subsubsection{Talk about some of the relationships you observed in this
part of the investigation. How did the feature(s) of interest vary with
other features in the
dataset?}\label{talk-about-some-of-the-relationships-you-observed-in-this-part-of-the-investigation.-how-did-the-features-of-interest-vary-with-other-features-in-the-dataset}}

Price had a surprisingly high amount of correlation with the diamond
size, even before transforming the features. An approximately linear
relationship was observed when price was plotted on a log scale and
carat was plotted with a cube-root transform. The scatterplot that came
out of this also suggested that there was an upper bound on the diamond
prices available in the dataset, since the range of prices for the
largest diamonds was much narrower than would have been expected, based
on the price ranges of smaller diamonds.

There was also an interesting relationship observed between price and
the categorical features. For all of cut, color, and clarity, lower
prices were associated with increasing quality. One of the potentially
major interacting factors is the fact that improved quality levels were
also associated with smaller diamonds. This will have to be explored
further in the next section.

\hypertarget{did-you-observe-any-interesting-relationships-between-the-other-features-not-the-main-features-of-interest}{%
\subsubsection{Did you observe any interesting relationships between the
other features (not the main feature(s) of
interest)?}\label{did-you-observe-any-interesting-relationships-between-the-other-features-not-the-main-features-of-interest}}

Expected relationships were found in the association between the `x',
`y', and `z' measurements of diamonds to the other linear dimensions as
well as to the `carat' variable. A small negative correlation was
observed between table size and depth, but neither of these variables
show a strong correlation with price, so they won't be explored further.
There was also a small interaction in the categorical quality features.
Diamonds of lower clarity appear to have slightly better cut and color
grades.

\hypertarget{multivariate-exploration}{%
\subsection{Multivariate Exploration}\label{multivariate-exploration}}

The main thing I want to explore in this part of the analysis is how the
three categorical measures of quality play into the relationship between
price and carat.

    \hypertarget{faceting-for-multivariate-data}{%
\paragraph{Faceting for Multivariate
Data}\label{faceting-for-multivariate-data}}

you saw how FacetGrid could be used to subset your dataset across levels
of a categorical variable, and then create one plot for each subset.
Where the faceted plots demonstrated were univariate before, you can
actually use any plot type, allowing you to facet bivariate plots to
create a multivariate visualization.

    using hist() function to draw Histogram for each column and use
figsize(,) parameter to show it obviously.

    \begin{tcolorbox}[breakable, size=fbox, boxrule=1pt, pad at break*=1mm,colback=cellbackground, colframe=cellborder]
\prompt{In}{incolor}{88}{\boxspacing}
\begin{Verbatim}[commandchars=\\\{\}]
\PY{c+c1}{\PYZsh{} Use this, and more code cells, to explore your data. Don\PYZsq{}t forget to add}
\PY{c+c1}{\PYZsh{}   Markdown cells to document your observations and findings.}
\PY{n}{df\PYZus{}go\PYZus{}bike}\PY{o}{.}\PY{n}{hist}\PY{p}{(}\PY{n}{figsize} \PY{o}{=} \PY{p}{(}\PY{l+m+mi}{20}\PY{p}{,}\PY{l+m+mi}{20}\PY{p}{)}\PY{p}{)}\PY{p}{;}
\end{Verbatim}
\end{tcolorbox}

    \begin{center}
    \adjustimage{max size={0.9\linewidth}{0.9\paperheight}}{Ford_Go_Bike_Part1_files/Ford_Go_Bike_Part1_145_0.png}
    \end{center}
    { \hspace*{\fill} \\}
    
    \hypertarget{using-plotting.scatter_matrix-function-to-draw-histogramsfor-each-column-and-scatter-plotting-between-numerical-columns-and-use-figsize-parameter-to-show-it-obviously}{%
\subparagraph{using plotting.scatter\_matrix() function to draw
Histogramsfor each column and scatter plotting between numerical columns
and use figsize(,) parameter to show it
obviously}\label{using-plotting.scatter_matrix-function-to-draw-histogramsfor-each-column-and-scatter-plotting-between-numerical-columns-and-use-figsize-parameter-to-show-it-obviously}}

    \begin{tcolorbox}[breakable, size=fbox, boxrule=1pt, pad at break*=1mm,colback=cellbackground, colframe=cellborder]
\prompt{In}{incolor}{ }{\boxspacing}
\begin{Verbatim}[commandchars=\\\{\}]
\PY{n}{pd}\PY{o}{.}\PY{n}{plotting}\PY{o}{.}\PY{n}{scatter\PYZus{}matrix}\PY{p}{(}\PY{n}{df\PYZus{}go\PYZus{}bike}\PY{p}{,} \PY{n}{figsize} \PY{o}{=} \PY{p}{(}\PY{l+m+mi}{30}\PY{p}{,}\PY{l+m+mi}{30}\PY{p}{)}\PY{p}{)}\PY{p}{;}
\end{Verbatim}
\end{tcolorbox}

    The faceted box plot suggests a slight interaction between the two
categorical variables, where, in level B of ``member\_gender'', the
level of ``user\_type'' seems to be have a larger effect on the value of
``bike\_id'', compared to the trend within ``member\_gender'' level A.

    \begin{tcolorbox}[breakable, size=fbox, boxrule=1pt, pad at break*=1mm,colback=cellbackground, colframe=cellborder]
\prompt{In}{incolor}{ }{\boxspacing}
\begin{Verbatim}[commandchars=\\\{\}]
\PY{n}{g} \PY{o}{=} \PY{n}{sns}\PY{o}{.}\PY{n}{FacetGrid}\PY{p}{(}\PY{n}{data} \PY{o}{=} \PY{n}{df\PYZus{}go\PYZus{}bike}\PY{p}{,} \PY{n}{col} \PY{o}{=} \PY{l+s+s1}{\PYZsq{}}\PY{l+s+s1}{member\PYZus{}gender}\PY{l+s+s1}{\PYZsq{}}\PY{p}{,} \PY{n}{size} \PY{o}{=} \PY{l+m+mi}{6}\PY{p}{)}
\PY{n}{g}\PY{o}{.}\PY{n}{map}\PY{p}{(}\PY{n}{sns}\PY{o}{.}\PY{n}{boxplot}\PY{p}{,} \PY{l+s+s1}{\PYZsq{}}\PY{l+s+s1}{user\PYZus{}type}\PY{l+s+s1}{\PYZsq{}}\PY{p}{,} \PY{l+s+s1}{\PYZsq{}}\PY{l+s+s1}{bike\PYZus{}id}\PY{l+s+s1}{\PYZsq{}}\PY{p}{)}
\end{Verbatim}
\end{tcolorbox}

    The faceted box plot suggests a slight interaction between the two
categorical variables, where, in level B of ``member\_birth\_year'', the
level of ``user\_type'' seems to be have a larger effect on the value of
``member\_gender'', compared to the trend within ``member\_birth\_year''
level A.

FacetGrid also allows for faceting a variable not just by columns, but
also by rows. We can set one categorical variable on each of the two
facet axes for one additional method of depicting multivariate trends.

    Setting margin\_titles = True means that instead of each facet being
labeled with the combination of row and column variable, labels are
placed separately on the top and right margins of the facet grid. This
is a boon, since the default plot titles are usually too long.

    \begin{tcolorbox}[breakable, size=fbox, boxrule=1pt, pad at break*=1mm,colback=cellbackground, colframe=cellborder]
\prompt{In}{incolor}{ }{\boxspacing}
\begin{Verbatim}[commandchars=\\\{\}]
\PY{n}{g} \PY{o}{=} \PY{n}{sns}\PY{o}{.}\PY{n}{FacetGrid}\PY{p}{(}\PY{n}{data} \PY{o}{=} \PY{n}{df\PYZus{}go\PYZus{}bike}\PY{p}{,} \PY{n}{col} \PY{o}{=} \PY{l+s+s1}{\PYZsq{}}\PY{l+s+s1}{user\PYZus{}type}\PY{l+s+s1}{\PYZsq{}}\PY{p}{,} \PY{n}{row} \PY{o}{=} \PY{l+s+s1}{\PYZsq{}}\PY{l+s+s1}{member\PYZus{}birth\PYZus{}year}\PY{l+s+s1}{\PYZsq{}}\PY{p}{,} \PY{n}{size} \PY{o}{=} \PY{l+m+mf}{2.5}\PY{p}{,}\PY{n}{margin\PYZus{}titles} \PY{o}{=} \PY{k+kc}{True}\PY{p}{)}
\PY{n}{g}\PY{o}{.}\PY{n}{map}\PY{p}{(}\PY{n}{plt}\PY{o}{.}\PY{n}{scatter}\PY{p}{,} \PY{l+s+s1}{\PYZsq{}}\PY{l+s+s1}{bike\PYZus{}id}\PY{l+s+s1}{\PYZsq{}}\PY{p}{,} \PY{l+s+s1}{\PYZsq{}}\PY{l+s+s1}{member\PYZus{}gender}\PY{l+s+s1}{\PYZsq{}}\PY{p}{)}
\end{Verbatim}
\end{tcolorbox}

    The code for the 2-d bar chart doesn't actually change much. The actual
heatmap call is still the same, only the aggregation of values changes.
Instead of taking size after the groupby operation, we compute the mean
across dataframe columns and isolate the column of interest.

    \begin{tcolorbox}[breakable, size=fbox, boxrule=1pt, pad at break*=1mm,colback=cellbackground, colframe=cellborder]
\prompt{In}{incolor}{ }{\boxspacing}
\begin{Verbatim}[commandchars=\\\{\}]
\PY{n}{cat\PYZus{}means} \PY{o}{=} \PY{n}{df\PYZus{}go\PYZus{}bike}\PY{o}{.}\PY{n}{groupby}\PY{p}{(}\PY{p}{[}\PY{l+s+s1}{\PYZsq{}}\PY{l+s+s1}{member\PYZus{}gender}\PY{l+s+s1}{\PYZsq{}}\PY{p}{,} \PY{l+s+s1}{\PYZsq{}}\PY{l+s+s1}{user\PYZus{}type}\PY{l+s+s1}{\PYZsq{}}\PY{p}{]}\PY{p}{)}\PY{o}{.}\PY{n}{mean}\PY{p}{(}\PY{p}{)}\PY{p}{[}\PY{l+s+s1}{\PYZsq{}}\PY{l+s+s1}{duration\PYZus{}hr}\PY{l+s+s1}{\PYZsq{}}\PY{p}{]}
\PY{n}{cat\PYZus{}means} \PY{o}{=} \PY{n}{cat\PYZus{}means}\PY{o}{.}\PY{n}{reset\PYZus{}index}\PY{p}{(}\PY{n}{name} \PY{o}{=} \PY{l+s+s1}{\PYZsq{}}\PY{l+s+s1}{duration\PYZus{}hr\PYZus{}avg}\PY{l+s+s1}{\PYZsq{}}\PY{p}{)}
\PY{n}{cat\PYZus{}means} \PY{o}{=} \PY{n}{cat\PYZus{}means}\PY{o}{.}\PY{n}{pivot}\PY{p}{(}\PY{n}{index} \PY{o}{=} \PY{l+s+s1}{\PYZsq{}}\PY{l+s+s1}{user\PYZus{}type}\PY{l+s+s1}{\PYZsq{}}\PY{p}{,} \PY{n}{columns} \PY{o}{=} \PY{l+s+s1}{\PYZsq{}}\PY{l+s+s1}{member\PYZus{}gender}\PY{l+s+s1}{\PYZsq{}}\PY{p}{,}
                            \PY{n}{values} \PY{o}{=} \PY{l+s+s1}{\PYZsq{}}\PY{l+s+s1}{duration\PYZus{}hr\PYZus{}avg}\PY{l+s+s1}{\PYZsq{}}\PY{p}{)}
\PY{n}{sns}\PY{o}{.}\PY{n}{heatmap}\PY{p}{(}\PY{n}{cat\PYZus{}means}\PY{p}{,} \PY{n}{annot} \PY{o}{=} \PY{k+kc}{True}\PY{p}{,} \PY{n}{fmt} \PY{o}{=} \PY{l+s+s1}{\PYZsq{}}\PY{l+s+s1}{.3f}\PY{l+s+s1}{\PYZsq{}}\PY{p}{,}
           \PY{n}{cbar\PYZus{}kws} \PY{o}{=} \PY{p}{\PYZob{}}\PY{l+s+s1}{\PYZsq{}}\PY{l+s+s1}{label}\PY{l+s+s1}{\PYZsq{}} \PY{p}{:} \PY{l+s+s1}{\PYZsq{}}\PY{l+s+s1}{mean(duration\PYZus{}hr)}\PY{l+s+s1}{\PYZsq{}}\PY{p}{\PYZcb{}}\PY{p}{)}
\end{Verbatim}
\end{tcolorbox}

    \hypertarget{talk-about-some-of-the-relationships-you-observed-in-this-part-of-the-investigation.-were-there-features-that-strengthened-each-other-in-terms-of-looking-at-your-features-of-interest}{%
\subsubsection{Talk about some of the relationships you observed in this
part of the investigation. Were there features that strengthened each
other in terms of looking at your feature(s) of
interest?}\label{talk-about-some-of-the-relationships-you-observed-in-this-part-of-the-investigation.-were-there-features-that-strengthened-each-other-in-terms-of-looking-at-your-features-of-interest}}

I extended my investigation of price against diamond size in this
section by looking at the impact of the three categorical quality
features. The multivariate exploration here showed that there indeed is
a positive effect of increased quality grade on diamond price, but in
the dataset, this is initially hidden by the fact that higher grades
were more prevalent in smaller diamonds, which fetch lower prices
overall. Controlling for the carat weight of a diamond shows the effect
of the other C's of diamonds more clearly. This effect was clearest for
the color and clarity variables, with less systematic trends for cut.

\hypertarget{were-there-any-interesting-or-surprising-interactions-between-features}{%
\subsubsection{Were there any interesting or surprising interactions
between
features?}\label{were-there-any-interesting-or-surprising-interactions-between-features}}

Looking back on the point plots, it doesn't seem like there's a
systematic interaction effect between the three categorical features.
However, the features also aren't fully independent. But it is
interesting in something like the 1-carat plot for prices against cut
and clarity, the shape of the `cut' dots is fairly similar for the SI2
through VVS2 clarity levels.

    \begin{tcolorbox}[breakable, size=fbox, boxrule=1pt, pad at break*=1mm,colback=cellbackground, colframe=cellborder]
\prompt{In}{incolor}{ }{\boxspacing}
\begin{Verbatim}[commandchars=\\\{\}]
\PY{o}{!}jupyter nbconvert \PYZhy{}\PYZhy{}to html \PYZhy{}\PYZhy{}no\PYZhy{}input Ford\PYZus{}Go\PYZus{}Bike\PYZus{}Part1.ipynb  
\end{Verbatim}
\end{tcolorbox}

    \begin{tcolorbox}[breakable, size=fbox, boxrule=1pt, pad at break*=1mm,colback=cellbackground, colframe=cellborder]
\prompt{In}{incolor}{ }{\boxspacing}
\begin{Verbatim}[commandchars=\\\{\}]
\PY{o}{!}jupyter nbconvert \PYZhy{}\PYZhy{}to script Ford\PYZus{}Go\PYZus{}Bike\PYZus{}Part1.ipynb
\end{Verbatim}
\end{tcolorbox}

    \begin{tcolorbox}[breakable, size=fbox, boxrule=1pt, pad at break*=1mm,colback=cellbackground, colframe=cellborder]
\prompt{In}{incolor}{ }{\boxspacing}
\begin{Verbatim}[commandchars=\\\{\}]
\PY{o}{!}jupyter nbconvert \PYZhy{}\PYZhy{}to latex Ford\PYZus{}Go\PYZus{}Bike\PYZus{}Part1.ipynb
\end{Verbatim}
\end{tcolorbox}

    \begin{tcolorbox}[breakable, size=fbox, boxrule=1pt, pad at break*=1mm,colback=cellbackground, colframe=cellborder]
\prompt{In}{incolor}{ }{\boxspacing}
\begin{Verbatim}[commandchars=\\\{\}]
\PY{o}{!}jupyter nbconvert \PYZhy{}\PYZhy{}to slides Ford\PYZus{}Go\PYZus{}Bike\PYZus{}Part1.ipynb
\end{Verbatim}
\end{tcolorbox}


    % Add a bibliography block to the postdoc
    
    
    
\end{document}
